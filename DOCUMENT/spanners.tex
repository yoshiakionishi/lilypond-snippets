%%%%%%%%%%%%%%%%%%%
%%%%%%%%%%%%%%%%%%%

\chapter {Spanners}

This chapter covers snippets that take advantages of spanners (text, line, etc.) in one way or another. Because functions such as \verb|\startTextSpan| and \verb|\stopTextSpan| activate and deactivate these snippets, caution must be paid when using more than one of them at the same time. See \hyperref[sec:comb_spanners]{Example in \textbf{Combinations}} to avoid conflicts between or among the spanner snippets.
\clearpage

\section {Arrow Lines}
\label{sec:arrow_lines}
\lilypondfile[staffsize=20,hpadding=4,voffset=4pt]{/arrow_lines.ly}
\hfill

\subsection{Description}
Implementation of arrow lines. It takes advantage of the \verb|\glissando| function. It is possible to have the arrow line span over multiple notes, as \verb|glissando-skip| parameter is set to \verb|##t|. When the arrow line spans over multiple systems, the arrow mark will not appaer at the end of the system.\footnote{See \hyperref[sec:arrow_lines_discussion]{\textbf{Discussion}} for more details.} Furthermore, it is possible to use the function on dyads and chords. The placement of the beginning of the arrow is adjusted according to the different types of notehead. 
\subsection{Grammar}
\begin{verbatim}
\arrowLineOn STARTING_NOTE (NOTES...)
\arrowLineOff ARRIVAL_NOTE
\end{verbatim}



\subsection{Code}
\begin{Verbatim}[numbers=left,xleftmargin=5mm]
\version "2.24.4"

arrowLineOn =
#(define-music-function (note)(ly:music?)
  (define paddingvalue (if (music-is-of-type? note 'event-chord)
                           (ly:duration-log
                            (ly:music-property
                             (first
                              (ly:music-property note 'elements))
                             'duration))
                           (ly:duration-log
                            (ly:music-property note 'duration))))

  #{

   \override Glissando.breakable = ##t
   \override Glissando.after-line-breaking = ##t
   \override Glissando.thickness = #2.35
   \override Glissando.bound-details.right.arrow = ##t
   \override Glissando.bound-details.right-broken.arrow = ##f
   \override Glissando.bound-details.right-broken.padding = #-1
   \override Glissando.bound-details.left.padding =
   #(cond ((= paddingvalue 0) 0.85)
          ((= paddingvalue 1) 0.65)
          ((>= paddingvalue 2) 0.65))

   \override Glissando.bound-details.right.padding = #0.25
   #note
   \glissando  \override NoteColumn.glissando-skip = ##t
  #})


arrowLineOff =
{
 \revert Glissando.breakable
 \revert Glissando.after-line-breaking
 \revert Glissando.thickness
 \revert Glissando.bound-details.right.arrow
 \revert Glissando.bound-details.right-broken.arrow
 \revert Glissando.bound-details.right-broken.padding
 \revert Glissando.bound-details.left.padding
 \revert Glissando.bound-details.right.padding
 \revert NoteColumn.glissando-skip
}



\score {

 {
  \override Score.TimeSignature.stencil = ##f

  \arrowLineOn
  c'1
  \arrowLineOff

  \arrowLineOn
  g'2
  \arrowLineOff

  \arrowLineOn
  c'4
  \arrowLineOff

  \arrowLineOn
  g'8 \noBeam
  \arrowLineOff
  \arrowLineOn
  c'16 s16 |

  \break
  \arrowLineOff
  \arrowLineOn
  c''8
  \override Voice.NoteHead.transparent = ##t
  8 8 8 8
  \revert Voice.NoteHead.transparent
  \arrowLineOff

  \arrowLineOn
  8 \harmonic
  \override Voice.NoteHead.transparent = ##t
  \once \override Voice.Dots.extra-offset = #'(-1 . -0.75)

  8.
  \revert Voice.NoteHead.transparent
  \arrowLineOff

  \arrowLineOn
  16

  \override Voice.NoteHead.transparent = ##t
  8 8 8 8
  \revert Voice.NoteHead.transparent
  \arrowLineOff
  \easyHeadsOn
  \arrowLineOn
  8
  \arrowLineOff
  \easyHeadsOff
  4.
  \break
  \arrowLineOn
  <c' a''>1

  \arrowLineOff
  <aes'' c'>1

  <<
   { \arrowLineOn a''1 \arrowLineOff c'1} \\
   {\arrowLineOn c'1 \arrowLineOff aes''1}
  >>

  \break
  \override Voice.Stem.stencil = ##f
  \override Voice.NoteHead.stencil = #ly:text-interface::print
  \override Voice.NoteHead.text =\markup{\musicglyph "noteheads.s1"}
  \set glissandoMap = #'((1 . 1) (1 . 1))
  \arrowLineOn
  <c' e' g' bes' d'' fis'' a''>2
  s4
  \arrowLineOff
  <c'
  \single \override NoteHead.text =
  \markup{\musicglyph "noteheads.s2"}  es'
  g' bes' d'' fis'' a''>4

 }


 \layout {

  indent = #0
  line-width = #125
  ragged-last = ##f

  \context {
   \Score
   proportionalNotationDuration = #(ly:make-moment 1/7)
  }
 }
}
\end{Verbatim}
\subsection{Discussion}
\label{sec:arrow_lines_discussion}
There are a few things to consider when using the arrow lines on dyads and chords:
\begin{itemize}
\item By default, all pair of notes will have arrow lines. In order to selectively show the arrow lines, use \verb|\set glissandoMap|. See \href{https://lilypond.org/doc/v2.24/Documentation/notation/expressive-marks-as-lines}{\textbf{1.3.3 Expressive marks as lines}} in LilyPond's \textit{Notation Reference} for details.
\item  Just as the ordinary \verb|\glissando| function, the X coordinate of the terminating point for \textit{all} of the lines between two dyads/chords is determined by the presence of accidentals in the arrival dyads/chords. Thus, if there is an accidental on one or more of the pitches in the arriving dyad/chord, there may be a space between the tip of arrow and the pitches \textit{without} the accidentals. Should this be avoided, it is best to apply the arrow lines in different layers, so that each of the layers will have a different X-coordinate value of the terminating point of the arrow lines.
\end {itemize}
\hyperref[sec:toc]{\textbf{Table of Contents}}

\vfill \break

%%%%%%%%%%%%%%%%%%%


\section {Boulez Tie}
\label{sec:arrow_lines}
\lilypondfile[staffsize=20,hpadding=4,voffset=4pt]{/boulez_tie.ly}
\hfill

\subsection{Description}
Implementation of ties seen in Movement 5 \textit{Tombeau} of \textit{Pli selon pli} by Pierre Boulez\autocite{RN1000}. The shape is roughly a combination of a short, \textit{laissez vibrer} sign on both originating and arriving notes, connected by a horizontal line. In the score, these are manually written in, and the shape is somewhat irregular. 
 
\subsection{Grammar}
\begin{verbatim}

\override Tie.stencil = \boulezTie
 
\end{verbatim}



\subsection{Code}
\begin{Verbatim}[numbers=left,xleftmargin=5mm]
\version "2.24.4"

\language "english"

\pointAndClickOff

boulezTie =
#(lambda (grob)

  (let (
        ;defines four control points, each of which containing
        ;x and y, defined from a to h...
        (a (caar (ly:grob-property grob 'control-points)) )
        (b  (cdar (ly:grob-property grob 'control-points)) )
        (c (car (cadr (ly:grob-property grob 'control-points))))
        (d (cdr (cadr (ly:grob-property grob 'control-points))))
        (e (car (caddr (ly:grob-property grob 'control-points))))
        (f (cdr (caddr (ly:grob-property grob 'control-points))))
        (g (car (cadddr (ly:grob-property grob 'control-points))))
        (h (cdr (cadddr (ly:grob-property grob 'control-points))))
        )
   (make-path-stencil

    (list 'moveto a b
          'curveto a b (* (+ a (+ c 0.5)) 0.5) d (+ c 0.5) b
          'lineto (- e 0.5) b
          'curveto (- e 0.5) b (+ (- e 0.5) (* (- g (- e 0.5)) 0.5)) f g h

          'curveto g h (+ (- e 0.5) (* (- g (- e 0.5)) 0.6))
          (+ f 0.1) (- e 0.5) (+ b 0.15)
          'lineto (+ c 0.5) (+ b 0.15)
          'curveto (+ c 0.5) (+ b 0.15) (* (+ a (+ c 0.5)) 0.5) (+ d 0.1) a b
          'closepath
          )

    0.01
    1
    1
    #t
    )
   )
  )

{

 \override Tie.stencil = \boulezTie

 c'1 ~ c'1
 c''' ~ c'''
 <a' d'' g''> ~ <a' d'' g''>
}}


\end{Verbatim}
\subsection{Discussion}
\label{sec:boulez_tie_discussion}
It is possible to adjust the shape of the tie using \verb|\shape| and \verb|\vshape| commands. 

\hyperref[sec:toc]{\textbf{Table of Contents}}

\vfill \break

%%%%%%%%%%%%%%%%%%%



\section {Dashed Arrow Lines}
\lilypondfile[staffsize=20,hpadding=4,voffset=4pt]{/dashed_arrow_lines.ly}
\hfill

\subsection{Description}
Implementation of dashed arrow lines. Its design is nearly identical to that of the \hyperref[sec:arrow_lines]{\textbf{Arrow Lines}}.
\subsection{Grammar}
\begin{verbatim}
\dashedArrowLineOn STARTING_NOTE (NOTES...)
\dashedArrowLineOff ARRIVAL_NOTE
\end{verbatim}

\subsection{Code}
\begin{Verbatim}[numbers=left,xleftmargin=5mm]
\version "2.24.4"

dashedArrowLineOn =
#(define-music-function (note)(ly:music?)
  (define paddingvalue (if (music-is-of-type? note 'event-chord)
                           (ly:duration-log
                            (ly:music-property
                             (first
                              (ly:music-property note 'elements))
                             'duration))
                           (ly:duration-log
                            (ly:music-property note 'duration))))

  #{

   \override Glissando.breakable = ##t
   \override Glissando.after-line-breaking = ##t
   \override Glissando.thickness = #2.35
   \override Glissando.style = #'dashed-line
   \override Glissando.bound-details.right.arrow = ##t
   \override Glissando.bound-details.right-broken.arrow = ##f
   \override Glissando.bound-details.right-broken.padding = #-1
   \override Glissando.bound-details.left.padding =
   #(cond ((= paddingvalue 0) 0.85)
          ((= paddingvalue 1) 0.65)
          ((>= paddingvalue 2) 0.65))

   \override Glissando.bound-details.right.padding = #0.25
   #note
   \glissando  \override NoteColumn.glissando-skip = ##t
  #})


dashedArrowLineOff =
{
 \revert Glissando.breakable
 \revert Glissando.after-line-breaking
 \revert Glissando.thickness
 \revert Glissando.style
 \revert Glissando.bound-details.right.arrow
 \revert Glissando.bound-details.right-broken.arrow
 \revert Glissando.bound-details.right-broken.padding
 \revert Glissando.bound-details.left.padding
 \revert Glissando.bound-details.right.padding
 \revert NoteColumn.glissando-skip
}



\score {

 {
  \override Score.TimeSignature.stencil = ##f

  \dashedArrowLineOn
  c'1
  \dashedArrowLineOff

  \dashedArrowLineOn
  g'2
  \dashedArrowLineOff

  \dashedArrowLineOn
  c'4
  \dashedArrowLineOff

  \dashedArrowLineOn
  g'8 \noBeam
  \dashedArrowLineOff
  \dashedArrowLineOn
  c'16 s16 |

  \break
  \dashedArrowLineOff
  \dashedArrowLineOn
  c''8
  \override Voice.NoteHead.transparent = ##t
  8 8 8 8
  \revert Voice.NoteHead.transparent
  \dashedArrowLineOff

  \dashedArrowLineOn
  8 \harmonic
  \override Voice.NoteHead.transparent = ##t
  \once \override Voice.Dots.extra-offset = #'(-1 . -0.75)

  8.
  \revert Voice.NoteHead.transparent
  \dashedArrowLineOff

  \dashedArrowLineOn
  16

  \override Voice.NoteHead.transparent = ##t
  8 8 8 8
  \revert Voice.NoteHead.transparent
  \dashedArrowLineOff
  \easyHeadsOn
  \dashedArrowLineOn
  8
  \dashedArrowLineOff
  \easyHeadsOff
  4.
  \break
  \dashedArrowLineOn
  <c' a''>1

  \dashedArrowLineOff
  <aes'' c'>1

  <<
   { \dashedArrowLineOn a''1 \dashedArrowLineOff c'1} \\
   {\dashedArrowLineOn c'1 \dashedArrowLineOff aes''1}
  >>
  \break
  \override Voice.Stem.stencil = ##f
  \override Voice.NoteHead.stencil = #ly:text-interface::print
  \override Voice.NoteHead.text =\markup{\musicglyph "noteheads.s1"}
  \set glissandoMap = #'((1 . 1) (1 . 1))
  \dashedArrowLineOn
  <c' e' g' bes' d'' fis'' a''>2
  s4
  \dashedArrowLineOff
  <c'
  \single \override NoteHead.text =
  \markup{\musicglyph "noteheads.s2"}  es'
  g' bes' d'' fis'' a''>4

 }

 \layout {

  indent = #0
  line-width = #125

  ragged-last = ##f

  \context {
   \Score
   proportionalNotationDuration = #(ly:make-moment 1/7)
  }
 }
}
\end{Verbatim}
\subsection{Discussion}
See \hyperref[sec:arrow_lines_discussion]{\textbf{Discussion} of the Arrow Lines}.
\hyperref[sec:toc]{\textbf{Table of Contents}}

\vfill \break

%%%%%%%%%%%%%%%%%%%





\section {Flat Slurs}
\hfill
\lilypondfile[staffsize=24,hpadding=4,voffset=4pt]{/flat_slur.ly}
\hfill

\subsection{Description}

Implementation of a straight-line slur in which only the beginning and ending parts are curved. This type of slurs can be found in scores by such composers as Brian Ferneyhough, Claus-Steffen Mahnkopf, and Donald Martino, where such slurs were manually drawn using a ruler. Even in the age of digital musical engraving, composers such as Stefan Beyer have emulated this slur using the musical engraving software programs.

See \hyperref[sec:flat_slurs_discussion]{\textbf{Discussion}} for more details.

\subsection{Grammar}
\begin{verbatim}
\override Slur.stencil = \flatSlur
\end{verbatim}

\subsection{Code}
\begin{Verbatim}[numbers=left,xleftmargin=5mm]
\version "2.24.4"

\language "english"

\pointAndClickOff

flatSlur =
#(lambda (grob)

  (let (
        ;defines four control points, each of which containing
        ;x and y, defined from a to h...
        (a (caar (ly:grob-property grob 'control-points)) )
        (b  (cdar (ly:grob-property grob 'control-points)) )
        (c (car (cadr (ly:grob-property grob 'control-points))))
        (d (cdr (cadr (ly:grob-property grob 'control-points))))
        (e (car (caddr (ly:grob-property grob 'control-points))))
        (f (cdr (caddr (ly:grob-property grob 'control-points))))
        (g (car (cadddr (ly:grob-property grob 'control-points))))
        (h (cdr (cadddr (ly:grob-property grob 'control-points))))
        )
   (make-path-stencil
    (if (= (ly:grob-property grob 'direction) 1)
        ;this is if the slur/tie direction is up...
        (list 'moveto a b
              'curveto a b a (- d 0.2) c (- d 0.2)
              'lineto e (- f 0.2)
              'curveto e (- f 0.2) g (- f 0.2) g h

              'curveto g h (+ g 0.1) (- f 0.5) e (- f 0.4)
              'lineto c (- d 0.4)
              'curveto c (- d 0.4) (+ a 0.1) (- d 0.3) a b
              'closepath
              )
        ;if the direction is down...
        (list 'moveto a b
              'curveto a b a (+ d 0.2) c (+ d 0.2)
              'lineto e (+ f 0.2)

              'curveto e (+ f 0.2) g (+ f 0.2) g h

              'curveto g h (- g 0.1) (+ f 0.5) e (+ f 0.4)
              'lineto c (+ d 0.4)
              'curveto c (+ d 0.4) (+ a 0.3) (+ d 0.3) a b
              'closepath
              ))

    0.01
    1
    1
    #t
    )
   )
  )
\layout {
 indent = 0
 line-width = 80
}
{
 \time 2/4
 \override Slur.stencil = \flatSlur

 \override Slur.details =
 #'(
    (max-slope . 0)
    (max-slope-factor . 100)
    (free-head-distance . 0.2)
    (free-slur-distance . 0)
    (extra-encompass-free-distance . 0.3)
    (extra-encompass-collision-distance . 0.8)
    (head-slur-distance-max-ratio . 3)
    (head-slur-distance-factor . 10)
    )

 \shape #'(
           (0 . 1)
           (0 . 0.5)
           (0 . 0.5)
           (0 . 0.25)
           ) Slur

 e'16 ( d' c' d' e' d' c'8 )

 \shape #'(
           (0 . 0)
           (0 . -0.5)
           (0 . -0.5)
           (0 . -1)
           ) Slur
 g''16 ( f'' e'' f'' g'' f'' e''8 ) \break



 c''4 ( b' a' g' ~ | \break

 g'8 )

 \shape #'(
           (0 . 0)
           (0 . 0.2)
           (0.75 . 2.5)
           (0 . 3)
           ) Slur


 a ( c' e' g' b' ) r4 |


}
\end{Verbatim}
\subsection{Discussion}
\label{sec:flat_slurs_discussion}
While this code can be used out of box by overriding \verb|Slur.stencil| to \verb|\flatSlur|, you may find that tweak will be needed. Initial experiments have shown that the following overrides of \verb|Slur.details| help:
\begin{Verbatim}[numbers=left,xleftmargin=5mm]
 \override Slur.details =
 #'(
    (max-slope . 0)
    (max-slope-factor . 100)
    (free-head-distance . 0.2)
    (free-slur-distance . 0)
    (extra-encompass-free-distance . 0.3)
    (extra-encompass-collision-distance . 0.8)
    (head-slur-distance-max-ratio . 3)
    (head-slur-distance-factor . 10)
    )
\end{Verbatim}

There may be other effective \verb|Slur.details| configurations.

Beyond this, further refinements can be made using \verb|\shape| and \verb|\vshape|.

\hyperref[sec:toc]{\textbf{Table of Contents}}

\vfill \break

%%%%%%%%%%%%%%%%%%%




\section {Grace Note Brackets I}
\hfill
\lilypondfile[staffsize=24,hpadding=4,voffset=4pt]{/gracenote_brackets.ly}
\hfill

\subsection{Description}
\textbf{NB}: See \hyperref[sec:gracenote2]{Grace Note Brackets II} for the updated version of this code.)
Replication of grace note brackets seen in scores by Pierre Boulez (e.g. \textit{Sur Incises}\autocite{RN1738}, \textit{...explosante-fixe...}\autocite{RN1737}). Bracket A in the example shows that the grace notes are to be played \textit{before} the beat to which they are applied. Whereas Bracket B shows that the grace notes are to be played \textit{on} the beat to which they are applied.

\subsection{Grammar}
\begin{verbatim}

\graceNoteBeforeBeatOn NOTE
\graceNoteBeforeBeatOff NOTE
\graceNoteAfterBeatOn NOTE
\graceNoteAfterBeatOff NOTE

\end{verbatim}

\subsection{Code}
\begin{Verbatim}[numbers=left,xleftmargin=5mm]
\version "2.24.4"

\language "english"

% This code includes snippet for grace note 
% slashes, which has been taken from:
% https://lsr.di.unimi.it/LSR/Item?id=1048


graceNoteBeforeBeatOn =
#(define-music-function (starting_note) (ly:music?)
   #{
     \once \override TextSpanner.style = #'line
     \once \override TextSpanner.bound-details.left.text =
     \markup { \draw-line #'(0 . -1) }
     \once \override TextSpanner.bound-details.right.text =
     \markup {
       \postscript
       "newpath 0 0 moveto 
0 -2.5 rlineto 
stroke 
newpath 
-0.275 -2 moveto 
0.275 -0.75 rlineto 
0.275 0.75 rlineto 
-0.275 -0.2 rlineto 
closepath 
fill"
     }
     \once \override TextSpanner.Y-offset = #5
     \once \override TextSpanner.bound-details.left.padding = #0.5
     \once \override TextSpanner.bound-details.right.padding = #-0.25
     #starting_note
     \startTextSpan
   #})


graceNoteBeforeBeatOff =
#(define-music-function (ending_note) (ly:music?)
   #{
     #ending_note
     \stopTextSpan
   #})


graceNoteAfterBeatOn =
#(define-music-function (starting_note) (ly:music?)
   #{
     \once \override TextSpanner.style = #'line
     \once \override TextSpanner.bound-details.right.text =
     \markup {
       \combine \draw-line #'(0 . -1)
       \postscript "newpath 
0 -1 moveto 
0 -1 rlineto 
stroke"
     }
     \once \override TextSpanner.bound-details.left.text =
     \markup {
       \postscript
       "newpath 0 0 moveto 
0 -1 rlineto 
stroke 
newpath 
-0.275 -0.75 moveto 
0.275 -0.75 rlineto 
0.275 0.75 rlineto 
-0.275 -0.2 rlineto 
closepath 
fill"
     }
     \once \override TextSpanner.Y-offset = #2
     \once \override TextSpanner.bound-details.left.padding = #0.5
     \once \override TextSpanner.bound-details.right.padding = #-0.25
     #starting_note
     \startTextSpan
   #})


graceNoteAfterBeatOff =
#(define-music-function (ending_note) (ly:music?)
   #{
     #ending_note
     \stopTextSpan
   #})

%%%%%%%%%%%%%%%%%%%LSR SNIPPET START%%%%%%%%%%%%%%%%%%

#(define (degrees->radians deg)
   (* PI (/ deg 180.0)))

slash =
#(define-music-function (ang stem-fraction protrusion)
   (number? number? number?)
   (remove-grace-property 'Voice 'Stem 'direction)
   #{
     \once \override Stem.stencil =
     #(lambda (grob)
        (let* ((x-parent (ly:grob-parent grob X))
               (is-rest? (ly:grob?
                          (ly:grob-object x-parent 'rest)))
               (beam (ly:grob-object grob 'beam))
               (stil (ly:stem::print grob)))
          (cond
           (is-rest? empty-stencil)
           ((ly:grob? beam)
            (let* ((refp (ly:grob-system grob))
                 (stem-y-ext (ly:grob-extent grob grob Y))
                 (stem-length
                  (- (cdr stem-y-ext) (car stem-y-ext)))
                 (beam-X-pos (ly:grob-property beam 'X-positions))
                 (beam-Y-pos (ly:grob-property beam 'positions))
                 (beam-slope (/ (- (cdr beam-Y-pos) (car beam-Y-pos))
                              (- (cdr beam-X-pos) (car beam-X-pos))))
                 (beam-angle (atan beam-slope))
                 (dir (ly:grob-property grob 'direction))
                 (line-dy (* stem-length stem-fraction))
                 (line-dy-with-protrusions (if (= dir 1)
                              (+ (* 4 protrusion) beam-angle)
                              (- (* 4 protrusion) beam-angle)))
                 (ang (if (> beam-slope 0)
                          (if (= dir 1)
                              (+ (degrees->radians ang) (* beam-angle 0.7))
                              (degrees->radians ang))
                          (if (= dir 1)
                              (degrees->radians ang)
                              (- (degrees->radians ang) (* beam-angle 0.7)))))
                 (line-dx (/ line-dy-with-protrusions (tan ang)))
                 (protrusion-dx (/ protrusion (tan ang)))
                 (corr (if (= dir 1) (car stem-y-ext) (cdr stem-y-ext))))
            (ly:stencil-add
             stil
             (grob-interpret-markup grob
                               (markup
                                 #:translate
                                 (cons (- protrusion-dx)
                                 (+ corr
                                    (* dir
                                       (- stem-length
                                          (+ stem-fraction protrusion)))))
                                 #:override '(thickness . 1.7)
                                 #:draw-line
                                 (cons line-dx
                                  (* dir line-dy-with-protrusions)))))))
           (else stil))))
   #})

startSlashedGraceMusic = {
  \slash 40 1 0.5
  \override Flag.stroke-style = #"grace"
}
stopSlashedGraceMusic = {
  \revert Flag.stroke-style
}

startAcciaccaturaMusic = {
  \slash 40 1 0.5
  s1*0(
  \override Flag.stroke-style = #"grace"
}
stopAcciaccaturaMusic = {
  \revert Flag.stroke-style
  s1*0)
}
%%%%%%%%%%%%%%%%%%% LSR SNIPPET END %%%%%%%%%%%%%%%%%%


{
  \grace {
    \startSlashedGraceMusic
    \graceNoteBeforeBeatOn d''8^\markup{\box A} e' f'' g' e'' c'
  }
  \graceNoteBeforeBeatOff d'2
  \grace {
    \startSlashedGraceMusic
    \graceNoteAfterBeatOn d''8^\markup{\box B} e' f'' g' e'' c'
  }
  \graceNoteAfterBeatOff d'2
}
\end{Verbatim}

\hyperref[sec:toc]{\textbf{Table of Contents}}

\vfill \break

%%%%%%%%%%%%%%%%%%%

\section {Grace Note Brackets II}
\label{sec:gracenote2}
\hfill
\lilypondfile[staffsize=24,hpadding=4,voffset=4pt]{/gracenote_brackets2.ly}
\hfill

\subsection{Description}
This is an updated version of Grace Note Brackets I. It differs from the original version in that this version takes a list of three parameters in order to finely adjust the shape of the bracket in order to accommodate various shapes of grace notes and the actual note. 

\subsection{Grammar}
\begin{verbatim}

\graceNoteBeforeBeatOn #'(OVERALL LEFT RIGHT) NOTE
\graceNoteBeforeBeatOff #'(OVERALL LEFT RIGHT) NOTE
\graceNoteAfterBeatOn #'(OVERALL LEFT RIGHT) NOTE
\graceNoteAfterBeatOff #'(OVERALL LEFT RIGHT) NOTE
\end{verbatim}
\textbf{NB}
The list accepts three integers as parameters, i.e.:
\begin{enumerate}
\item \verb|OVERALL| is a value of the distance between the top line of the staff and the horizontal line of the grace note bracket. This value cannot be smaller than the skyline value established by the staff line and the notes; when the skyline value is greater than what is specified in this bracket, the skyline value is favored. When in doubt, start with \verb|0|, then increase the amount gradually.
\item \verb|LEFT| and \verb|RIGHT| values (negative value only!)  adjust the lengths of the left and right hooks.   
\end{enumerate}
\subsection{Code}
\begin{Verbatim}[numbers=left,xleftmargin=5mm]
\version "2.24.4"
\language "english"

% This code includes snippet for grace note
% slashes, which has been taken from:
% https://lsr.di.unimi.it/LSR/Item?id=1048

% Slightly revised, Jan. 19/22 2025 - YO

graceNoteBeforeBeatOn =
#(define-music-function (setting-list starting_note) (list? ly:music? )
  #{
   \once \override TextSpanner.style = #'line
   \once \override TextSpanner.bound-details.left.text =
   \markup {
    \combine
    \draw-line #(cons 0 -0.5)
    \postscript #(string-append "newpath 
0 -0.5 moveto 
0 " (number->string (cadr setting-list)) " rlineto 
stroke")
   }
   \once \override TextSpanner.bound-details.right.text =
   \markup {
    \postscript
    #(string-append "newpath 0 0 moveto 
0 " (number->string (caddr setting-list))  " rlineto 
stroke 
newpath 
-0.275 " (number->string (+ (caddr setting-list) 0.25))  " moveto 
0.275 -0.75 rlineto 
0.275 0.75 rlineto 
-0.275 -0.2 rlineto 
closepath 
fill")
   }
   \once \override TextSpanner.extra-offset = #(cons 0 (car setting-list))
   \once \override TextSpanner.bound-details.left.padding = #0.5
   \once \override TextSpanner.bound-details.right.padding = #-0.25
   #starting_note
   \startTextSpan
  #})


graceNoteBeforeBeatOff =
#(define-music-function (ending_note) (ly:music?)
  #{
   #ending_note
   \stopTextSpan
  #})


graceNoteAfterBeatOn =
#(define-music-function (setting-list starting_note) (list? ly:music?)
  #{
   \once \override TextSpanner.style = #'line
   \once \override TextSpanner.bound-details.right.text =
   \markup {
    \combine
    \draw-line #(cons 0 -1)
    \postscript #(string-append "newpath 
0 -1 moveto 
0 " (number->string (caddr setting-list)) " rlineto 
stroke")
   }
   \once \override TextSpanner.bound-details.left.text =
   \markup {
    \postscript
    #(string-append "newpath 0 0 moveto 
0 " (number->string (cadr setting-list))  " rlineto 
stroke 
newpath 
-0.275 " (number->string (+ (cadr setting-list) 0.25))  " moveto 
0.275 -0.75 rlineto 
0.275 0.75 rlineto 
-0.275 -0.2 rlineto 
closepath 
fill")
   }
   \once \override TextSpanner.extra-offset = #(cons 0 (car setting-list))
   \once \override TextSpanner.bound-details.left.padding = #0.5
   \once \override TextSpanner.bound-details.right.padding = #-0.25
   #starting_note
   \startTextSpan
  #})


graceNoteAfterBeatOff =
#(define-music-function (ending_note) (ly:music?)
  #{
   #ending_note
   \stopTextSpan
  #})
  
%%%%%%%%%%%%%%%%%%%LSR SNIPPET START%%%%%%%%%%%%%%%%%%

#(define (degrees->radians deg)
  (* PI (/ deg 180.0)))

slash =
#(define-music-function (ang stem-fraction protrusion)
  (number? number? number?)
  (remove-grace-property 'Voice 'Stem 'direction)
  #{
   \once \override Stem.stencil =
   #(lambda (grob)
     (let* ((x-parent (ly:grob-parent grob X))
            (is-rest? (ly:grob?
                       (ly:grob-object x-parent 'rest)))
            (beam (ly:grob-object grob 'beam))
            (stil (ly:stem::print grob)))
      (cond
       (is-rest? empty-stencil)
       ((ly:grob? beam)
        (let* ((refp (ly:grob-system grob))
               (stem-y-ext (ly:grob-extent grob grob Y))
               (stem-length
                (- (cdr stem-y-ext) (car stem-y-ext)))
               (beam-X-pos (ly:grob-property beam 'X-positions))
               (beam-Y-pos (ly:grob-property beam 'positions))
               (beam-slope (/ (- (cdr beam-Y-pos) (car beam-Y-pos))
                              (- (cdr beam-X-pos) (car beam-X-pos))))
               (beam-angle (atan beam-slope))
               (dir (ly:grob-property grob 'direction))
               (line-dy (* stem-length stem-fraction))
               (line-dy-with-protrusions (if (= dir 1)
                                             (+ (* 4 protrusion) beam-angle)
                                             (- (* 4 protrusion) beam-angle)))
               (ang (if (> beam-slope 0)
                        (if (= dir 1)
                            (+ (degrees->radians ang) (* beam-angle 0.7))
                            (degrees->radians ang))
                        (if (= dir 1)
                            (degrees->radians ang)
                            (- (degrees->radians ang) (* beam-angle 0.7)))))
               (line-dx (/ line-dy-with-protrusions (tan ang)))
               (protrusion-dx (/ protrusion (tan ang)))
               (corr (if (= dir 1) (car stem-y-ext) (cdr stem-y-ext))))
         (ly:stencil-add
          stil
          (grob-interpret-markup grob
                                 (markup
                                  #:translate
                                  (cons (- protrusion-dx)
                                        (+ corr
                                           (* dir
                                              (- stem-length
                                                 (+ stem-fraction protrusion)))))
                                  #:override '(thickness . 1.7)
                                  #:draw-line
                                  (cons line-dx
                                        (* dir line-dy-with-protrusions)))))))
       (else stil))))
  #})

startSlashedGraceMusic = {
 \slash 40 1 0.5
 \override Flag.stroke-style = #"grace"
}
stopSlashedGraceMusic = {
 \revert Flag.stroke-style
}

startAcciaccaturaMusic = {
 \slash 40 1 0.5
 s1*0(
 \override Flag.stroke-style = #"grace"
}
stopAcciaccaturaMusic = {
 \revert Flag.stroke-style
 s1*0)
}
%%%%%%%%%%%%%%%%%%% LSR SNIPPET END %%%%%%%%%%%%%%%%%%

{
 \grace {
  \startSlashedGraceMusic
  \graceNoteBeforeBeatOn #'(1 -2 -1) d'8^\markup{\translate #'(0 . 3) \box A} 
  e f g e' c'
 }
 \graceNoteBeforeBeatOff d''2
 \grace {
  \startSlashedGraceMusic
  \graceNoteAfterBeatOn #'(0 -1 -2) d''8^\markup{\box B} e' f'' g' e'' c'
 }
 \graceNoteAfterBeatOff d'2
}
\end{Verbatim}

\hyperref[sec:toc]{\textbf{Table of Contents}}

\vfill \break

%%%%%%%%%%%%%%%%%%%


\section {Tempo Arrows}
\hfill
\lilypondfile[staffsize=24,hpadding=4,voffset=4pt]{/tempoArrows.ly}
\hfill

\subsection{Description}

Replication of accelerando and rallentando arrows chiefly seen in scores by Tōru Takemitsu.\footnote{Examples abound, but see: \cite{RN1736} and \cite{RN1735} Other composers from the same publishing company, e.g. Toshio Hosokawa, have also adopted variants of the arrows in their music.} The snippets also handle line break.

\subsection{Grammar}
\begin{verbatim}
\accelArrow #Line_angle ... \stopTextSpan
\rallArrow #Line_angle ... \stopTextSpan
\end{verbatim}


\textbf{NB} 
\begin{enumerate}
\item \verb|#Line_angle| sets how angled the horizontal line should be. \verb|#5| should be more than sufficient for a short line.  When it goes over a line break or it extends for a long time, a smaller number may be recommended, such as \verb|#2|. 
\item These commands only set the tempo arrows; as such, indications such as \verb|accel.| and \verb|rall.| need to be added separately. 
\end{enumerate}

\subsection{Code}
\begin{Verbatim}[numbers=left,xleftmargin=5mm]
\version "2.24.4"

% freely modified from: https://lsr.di.unimi.it/LSR/Item?id=1168
% as well as http://lsr.di.unimi.it/LSR/Item?id=1023


accelArrow =
#(define-music-function (line_angle) (number?)

   (define x_value (cos (* (/ 3.14159265358979 180) (- 90 line_angle))))
   (define y_value (sin (* (/ 3.14159265358979 180) (- 90 line_angle))))
   #{
     \tweak direction #up
     \tweak style  #'line
     \tweak thickness  #1
     \tweak to-barline ##t
     \tweak rotation #(list line_angle -1 0 )
     \tweak bound-details.left.stencil #ly:text-interface::print
     \tweak bound-details.left.text \markup \postscript
     #(string-append
       "gsave newpath 
0 0 moveto "
       (number->string x_value) " "
       (number->string y_value)
       " rlineto 
stroke 
grestore")
     \tweak bound-details.left-broken.stencil #ly:text-interface::print
     \tweak bound-details.left-broken.text ##f

     \tweak bound-details.right.stencil #ly:text-interface::print
     \tweak bound-details.right.text \markup \postscript
     "newpath 
0 0 moveto 
-1 -0.3 rlineto 
stroke"
     \tweak bound-details.right-broken.stencil #ly:text-interface::print
     \tweak bound-details.right-broken.text ##f
     \tweak font-shape  #'upright
     \tweak bound-details.left.padding  #0
     \tweak bound-details.right.padding  #0
     \tweak breakable ##t
     \tweak after-line-breaking ##t

     \startTextSpan
   #})

rallArrow =
#(define-music-function (line_angle) (number?)

   (define x_value (cos (* (/ 3.14159265358979 180) (- 90 line_angle))))
   (define y_value (sin (* (/ 3.14159265358979 180) (- 90 line_angle))))
   #{
     \tweak direction #up
     \tweak style  #'line
     \tweak thickness  #1
     \tweak to-barline ##t
     \tweak rotation #(list (* -1 line_angle) 1 0 )
     \tweak bound-details.left.stencil #ly:text-interface::print
     \tweak bound-details.left.text \markup \postscript
     #(string-append
       "gsave 
newpath 
0 0 moveto "
       (number->string x_value) " "
       (number->string (* -1 y_value))
       " rlineto 
stroke 
grestore")
     \tweak bound-details.left-broken.stencil #ly:text-interface::print
     \tweak bound-details.left-broken.text ##f

     \tweak bound-details.right.stencil #ly:text-interface::print
     \tweak bound-details.right.text \markup \postscript
     "newpath 
0 0 moveto 
-1 -0.3 rlineto 
stroke"
     \tweak bound-details.right-broken.stencil #ly:text-interface::print
     \tweak bound-details.right-broken.text ##f
     \tweak font-shape  #'upright
     \tweak bound-details.left.padding  #0
     \tweak bound-details.right.padding  #0
     \tweak breakable ##t
     \tweak after-line-breaking ##t

     \startTextSpan
   #})

\score {
  \layout {
    indent = 0
  }
  {
    c'2^\markup{\translate #'(-4 . 2) \box "A"}
    ^\markup {\translate #'(0 . 1.5) \tiny \bold "accel."} 
    	\accelArrow #5   c'2
    c'2 \after 2 \stopTextSpan c'2
    c'2 ^\markup {\translate #'(0 . 1.5) \tiny \bold "rit."}  
    	\rallArrow #3   c'2
    c'2 \after 2 \stopTextSpan c'2 \bar "||"
  }
}

\score {
  \layout {
    indent = 0
    line-width = 40
  }
  {
    c'2^\markup{\translate #'(-4 . 2) \box "B"}
    ^\markup {\translate #'(0 . 1.5) \tiny \bold "accel."} 
    	\accelArrow #5 c'2
    c'2 c'2
    c'2^\markup {\translate #'(0 . 1.5) \teeny \bold "(accel.)"} 
    	\after 2 \stopTextSpan c'2
    c'2 ^\markup {\translate #'(0 . 1.5) \tiny \bold "rit."}  
    	\rallArrow #2 c'2 \break
    c'2^\markup {\translate #'(0 . 1.5) \teeny \bold "(rit.)"} c'2
    c'2 \after 2 \stopTextSpan c'2 \bar "||"
  }
}
\end{Verbatim}

\hyperref[sec:toc]{\textbf{Table of Contents}}

\vfill \break

%%%%%%%%%%%%%%%%%%%

