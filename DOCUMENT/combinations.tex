%%%%%%%%%%%%%%%%%%%
%%%%%%%%%%%%%%%%%%%

\chapter {Combinations}

This chapter presents examples that combine several snippets from the previous chapters. \textbf{Variables Used} provides a comprehensive list of all the variables required to generate the snippet. Among these, indented variables indicate "variables of a variable," i.e., dependent variables necessary for the main variables to function. The \textbf{Code} section only lists the score portion of the LilyPond code.

\section {Prescriptive Notation for String Instruments}
\label{sec:comb_strings}
\hfill
\lilypondfile[staffsize=24,hpadding=4,voffset=4pt]{/pres_notation_strings.ly}
\hfill

\subsection{Description}
An example of a prescriptive notation for a string instrument. Vertical placement of the notehead corresponds to the position at which bowing takes place. Horizontally it shows the change of the bow pressure against the string(s).
\subsection{Variables Used}
\begin{verbatim}

\stringPositionClef
		\stringPositionClefDesign
\dashedLineNotehead
\modularLineNotehead
\noteheadless
\end{verbatim}
\subsection{Code}
\begin{Verbatim}[numbers=left,xleftmargin=5mm]

\score {
  {
    \override Staff.StaffSymbol.line-positions = #'(6  -6)
    \stringPositionClef 
    \dashedLineNotehead g'4
    	^\markup {\fontsize #-4 \italic flaut.}
    	^\markup \translate #'(-2.5 . -0) \center-column 
			{\translate #'(0 . -1.5) \fontsize #-4 II 
			\fontsize #-4 III} 
		a' #6
    \modularLineNotehead a'
    	^\markup \column {\translate #'(0 . -1.5) 
			\fontsize #-4 \italic apply \fontsize #-4 
			\italic pressure} 
		d'' #15 #150 #6
    \override TupletNumber.text = #tuplet-number::calc-fraction-text
    \stemUp  \tuplet 5/4 {
      \modularLineNotehead d''8 b' #150 #50 #2.5
      \modularLineNotehead b' f'' #50 #175 #2.5
      \modularLineNotehead f'' a' #175 #70 #2.5
      \modularLineNotehead a' c'' #70 #120 #2.5
      \modularLineNotehead c'' e' #120 #15 #3.5
    }
    |
    \modularLineNotehead e'4
    	^\markup {\fontsize #-4 \italic ord.} 
		e' #15 #15 #12
    \noteheadless e'
    \dashedLineNotehead e'
    	^\markup {\fontsize #-4 \italic (flaut.)} 
		e' #5
  }

  \layout {
    \context {
      \Score proportionalNotationDuration = #(ly:make-moment 1/10)    
      		\override SpacingSpanner.uniform-stretching = ##t
    }
  }
}

\end{Verbatim}

\hyperref[sec:toc]{\textbf{Table of Contents}}

\vfill \break

%%%%%%%%%%%%%%%%%%%

\section {Multiple Instances Of Spanners At Once}
\label{sec:comb_spanners}
\hfill
\lilypondfile[staffsize=24,hpadding=4,voffset=4pt]{/comb_spanners.ly}
\hfill

\subsection{Description}
Invoking two or more Text Spanners (that require \verb|\stopTextSpan| for them to finish their processes) all on one single layer could cause the spanners to behave unexpectedly. This entry is an attempt to avoid such unexpected behaviors by invoking a spanner per layer (A), or per staff line (B).

\subsection{Variables Used}
\begin{verbatim}
\startSlashedGraceMusic
\stopSlashedGraceMusic
\graceNoteBeforeBeatOn
\graceNoteBeforeBeatOff
\graceNoteAfterBeatOn
\graceNoteAfterBeatOff
\rallArrow

\end{verbatim}
\subsection{Code}
\begin{Verbatim}[numbers=left,xleftmargin=5mm]

%%%%%%%%%%%%%%%%%%% A %%%%%%%%%%%%%%%%%%
\score {
  \new Staff = "allElementsCombined"
  \with {instrumentName = \markup {\fontsize #4 \box "A"}}  {
    \numericTimeSignature
    \override Score.MetronomeMark.Y-offset = #5.75
    \tempo 4 = 100
    \time 5/4
    <<
      {
        \tieNeutral \stemNeutral d'4~
        \tuplet 3/2 {d'8 d'4}
        \stemUp \grace  {
          \startSlashedGraceMusic \graceNoteBeforeBeatOn e'8 f''
          \stopSlashedGraceMusic
        } \graceNoteBeforeBeatOff g'4~
        \stemNeutral g'8.[ \grace { 
        \startSlashedGraceMusic \graceNoteAfterBeatOn
          e''16 c'' e' c' \stopSlashedGraceMusic
        }
        \graceNoteAfterBeatOff d''16]~
        \tuplet 3/2 {d''8 d'8 d'8~} |
        \time 4/4
        d'1 \bar"||"
      }
      \\
      {
        s4  \tuplet 3/2 {
          s8 \override Voice.TextSpanner.Y-offset = #6.5
          s4^\markup {\translate #'(0 . 6.5) \bold "rall."}
          \rallArrow #4
        } s2. \tempo 4 = 50 s4*4 \stopTextSpan
      }
    >>
  }
}





%%%%%%%%%%%%%%%%%%% B %%%%%%%%%%%%%%%%%%
\score {
  <<
    \new Staff = "tempoLine" \with {
      \remove Clef_engraver
      \remove Staff_symbol_engraver
      \remove Time_signature_engraver
    }
    {
      \numericTimeSignature
      \override Score.MetronomeMark.Y-offset = #6
      \tempo 4 = 100
      \time 5/4
      s4  \tuplet 3/2 {
        s8 \override Voice.TextSpanner.Y-offset = #-2.25
        s4^\markup {\translate #'(0 . 0) \bold "rall."}
        \rallArrow #4} s2 \after 64*15 \stopTextSpan s8*2 |
      \tempo 4 = 50  s4*4
    }
    \new Staff = "music"
    \with { instrumentName = \markup {\fontsize #4 \box "B"}}
    {
      \tieNeutral \stemNeutral d'4~
      \tuplet 3/2 {d'8 d'4}
      \grace {
        \startSlashedGraceMusic \graceNoteBeforeBeatOn e'8 f''
        \stopSlashedGraceMusic
      } \graceNoteBeforeBeatOff g'4~
      g'8.[ \grace { \startSlashedGraceMusic \graceNoteAfterBeatOn
        e''16 c'' e' c' \stopSlashedGraceMusic
      }
      \graceNoteAfterBeatOff d''16]~
      \tuplet 3/2 {d''8 d'8 d'8~} |
      \time 4/4
      d'1 \bar"||"
    }
  >>
}
\end{Verbatim}

\hyperref[sec:toc]{\textbf{Table of Contents}}

\vfill \break

%%%%%%%%%%%%%%%%%%%
