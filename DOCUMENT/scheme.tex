%%%%%%%%%%%%%%%%%%%
%%%%%%%%%%%%%%%%%%%


\chapter {Exploring Scheme}
\VerbatimFootnotes
\section {Introduction}
Scheme, one of the dialects of the Lisp family of programming languages, is used in LilyPond as its extension language. Scheme allows LilyPond users to explore the inner workings of the program, enabling significant customization. The snippets in this document would not exist without taking advantage of it.\footnote{For newcomers: parts of LilyPond code written in Scheme are often enclosed in \verb+#(+ and \verb+)+. Numerical values preceded by \verb+#+, and number pairs such as \verb+\#'(1 . -2)+, are also part of the Scheme language.} 

However, learning Scheme can be daunting. In his unfinished book on Scheme and LilyPond, Urs Liska refers to its "thorny path."\autocite{RN1739} While I have experience with Common Lisp (another Lisp dialect) from my work with OpenMusic, adjusting to Scheme’s grammatical nuances still took some time.

This chapter does not aim to be a comprehensive guide to using Scheme in LilyPond.\footnote{For a deeper dive, refer to the resource by Liska, as well as \cite{RN1740}. LilyPond also provides its own Extending Manual: \url{https://lilypond.org/doc/v2.24/Documentation/extending/index}} Instead, it offers suggestions for newcomers to familiarize themselves with Scheme.

\subsection{Step 1a: Focus on the Scheme Language Itself}
Scheme is a language distinct from LilyPond, and understanding this distinction is essential. For simpler LilyPond tasks, Scheme may not be necessary. However, when working with internal parameters, Scheme allows deeper customization. It is beneficial to first study Scheme independently, learning its syntax and concepts by writing simple code.

\subsection{Step 1b: Get Used to Prefix Notation}
Scheme, like its Lisp relatives, uses prefix notation (Cambridge Polish Notation). Here are examples:

\begin{Verbatim}
(+ 12 34)    
>> This expression results in the value of 46.

(+ 4 (* 3 9))
>> This expression first resolves the multiplication: (+ 4 27), which is 31.
\end{Verbatim}

If you are new to this, I recommend starting with \cite{LS0001}. While you might be eager to dive into using Scheme in LilyPond, learning Scheme as a programming language will make the process smoother.\footnote{Liska and Samra’s resources serve as excellent refreshers later on.}

\subsection{Step 2: Study Lots of Snippets}
Once familiar with Scheme, study how it integrates with LilyPond by reviewing snippets from LSR. Start with shorter examples and analyze their structure. Here is an example snippet for adding the \textit{Schleifer} ornament:\footnote{\url{https://lsr.di.unimi.it/LSR/Item?id=1185}}
	\begin{figure}[!htb]
	\centering
	\lilypondfile[staffsize=16]{/LSR1185.ly}
	\caption{\label{fig:figure1} LSR No. 1185: \textit{Schleifer} Ornament.}
	\end{figure}
The corresponding code:\footnote{The mailing list thread referenced in the preamble is available at \url{https://lists.gnu.org/archive/html/lilypond-user/2021-09/msg00352.html}}
\begin{Verbatim}[numbers=left,xleftmargin=5mm]
% Implementation by Martin Straeten of the Schleifer ornament 
% as used by Johann Sebastian Bach, contributed to the user 
% mailing list. In this case, it functions like a set of (always?)
% two grace notes, hence using a modified grace note to represent 
% it in LilyPond makes sense.
%
% Code styling and user interface by Simon Albrecht 2024.

schleiferMarkup = \markup {
  \large \halign #.2 \raise #0.0
  \combine
  \halign #.8 \musicglyph "scripts.prall"
  \rotate #140 \normalsize \raise #2.4 \musicglyph "flags.u3"
}
schleiferGrace =
#(define-music-function (note) (ly:music?)
   #{
     \grace {
       \once\override NoteHead.stencil = #ly:text-interface::print
       \once\override NoteHead.X-extent = #'(-2 . -0)
       \once\override NoteHead.text = \schleiferMarkup
       \once\omit Stem
       \once\omit Flag
       $note
     }
   #})

\relative {
  \time 3/8
  \partial 8
  \clef bass
  \key c \minor
  g8
  \schleiferGrace c es8. d16 c8
  c4
}
\addlyrics {
  Ich ha -- be ge -- nug
}
\end{Verbatim}

The \verb|\schleiferGrace| variable creates a customized ornament using Scheme's \verb|define-music-function| macro. For a deeper understanding of the macro syntax, refer to the \textit{LilyPond – Internals Reference}.\footnote{\url{https://lilypond.org/doc/v2.24/Documentation/internals/scheme-functions}} \par
Taking the variable \verb|\schleiferGrace|, we see that invoking it creates an instance of activating a Scheme function that starts at Line 16. \verb|define-music-function| is a macro that allows you to create a function that operates on LilyPond. \par
According to \textit{LilyPond – Internals Reference}, the syntax for \verb|define-music-function| is: \par
\begin{verbatim}
(define-music-function (arg1 arg2 …)
                       (type1? type2? …)
    function-body)
\end{verbatim}

In the code, the argument's name is \verb|note|, and it is tested according to the type specified in \verb|type1?|, which in this case is \verb|ly:music?|. According to the \textit{Internal Reference}, \verb|ly:music?| is a function that checks whether the object—in this case, \verb|note|—is a \verb|Music| object. Thus, it becomes clear that this function will not work unless it is followed by a musical note. 

From Line 17 to Line 26, we see that a LilyPond code snippet has been inserted, as \verb+#{+ and \verb+#}+ signify the boundary of the LilyPond code within the Scheme code. This means that as part of invoking the variable \verb|\schleiferGrace|, it passes through this LilyPond fragment, which is responsible for creating a grace note. Here, the notehead of the grace note is replaced with \verb|\schleiferMarkup|, which is defined in Lines 9 to 14 of the code.\footnote{The technique of sequential overrides, invoking the Scheme command \verb+#ly:text-interface::print+, sets the \verb|.stencil| of the notehead to use whatever is defined in the \verb|.text| parameter. This technique is frequently used and is very useful in customizing notation. See also: \url{https://lilypond.org/doc/v2.24/Documentation/notation/modifying-stencils}.} 

Lines 22 and 23 show that the stem and flag are omitted from the grace note, while Line 24's \verb+$note+ signifies that the original argument \verb|note| is called upon.\footnote{Refer to this page for the difference between \verb+#+ and \verb+$+: \url{https://lilypond.org/doc/v2.24/Documentation/extending/lilypond-scheme-syntax}} In this way, the \textit{Schleifer} ornament is created from a note that follows the variable \verb|\schleiferGrace|. This note is transformed into a grace note with a customized stencil setting, all done within the Scheme code. 

\subsection {Step 3: Hack the Codes}

Once you study a code and become familiar with how it operates, experimenting with the code by hacking is a good way to deepen your understanding. Below, I give one example using the preceding \textit{Schleifer} ornament snippet.

The \textit{LilyPond -- Internal Reference} reveals that the object \verb+NoteHead+ has its own standard settings, as well as support for about a dozen other interfaces.\footnote{\url{https://lilypond.org/doc/v2.24/Documentation/internals/notehead}} One of them is the \verb+grob-interface+, which makes it possible to change the color of a graphical object, or \textit{Grob}.\footnote{\url{https://lilypond.org/doc/v2.24/Documentation/internals/grob_002dinterface}} Further reading in the \textit{LilyPond -- Notation Reference} shows that it is possible to override the color of an object.\footnote{\url{https://lilypond.org/doc/v2.24/Documentation/notation/inside-the-staff#coloring-objects}} Let us now tweak the \textit{Schleifer} ornament code to allow us to change the ornament's color.

Following the reference, add the following line underneath \verb+\once\override NoteHead.X-extent+:

\begin{verbatim}
\once\override NoteHead.color = #red
\end{verbatim}

Running LilyPond now should produce the following result:
	\begin{figure}[!htb]
	\centering
\lilypondfile[staffsize=16]{/LSR1185e1.ly}
	\caption{\label{fig:figure2} LSR No. 1185: \textit{Schleifer} Ornament in red.}
	\end{figure}
Hard-coding a change like this may be good for testing the waters, but we may want the \textit{Schleifer} ornament in more than just one color. The beauty of extending LilyPond is that we can customize the Scheme code to allow for this flexibility.

Let us move on. We should now let the \verb+define-music-function+ know that we are adding an additional argument to specify the color. The first part of the code will look like this:

\begin{verbatim}
#(define-music-function (note schleiferColor) (ly:music? color?) 
\end{verbatim}

This adds the argument \verb|schleiferColor|, which only accepts color, as indicated by the corresponding test function \verb|color?|.

Then, implement this argument in the sequence of \verb+\once\override+ processes. The line \verb+NoteHead.color+ can now be changed to:

\begin{verbatim}
\once\override NoteHead.color = #schleiferColor
\end{verbatim}

Now, the variable \verb+\schleiferGrace+ requires one more argument to specify the ornament's color. The entire code should look like this:

\begin{Verbatim}[numbers=left,xleftmargin=5mm]
schleiferMarkup = \markup {
  \large \halign #.2 \raise #0.0
  \combine
  \halign #.8 \musicglyph "scripts.prall"
  \rotate #140 \normalsize \raise #2.4 \musicglyph "flags.u3"
}

schleiferGrace =
#(define-music-function (note schleiferColor) (ly:music? color?)
   #{
     \grace {
       \once\override NoteHead.stencil = #ly:text-interface::print
       \once\override NoteHead.X-extent = #'(-2 . 0)
       \once\override NoteHead.color = #schleiferColor
       \once\override NoteHead.text = \schleiferMarkup
       \once\omit Stem
       \once\omit Flag
       $note
     }
   #})
\relative {
  \time 3/8
  \partial 8
  \clef bass
  \key c \minor
  g8
  \schleiferGrace c #green es8. d16 c8
  c4
}
\addlyrics {
  Ich ha -- be ge -- nug
}
\end{Verbatim}

This produces the following output:

	\begin{figure}[!htb]
	\centering
\lilypondfile[staffsize=16]{/LSR1185e2.ly}
	\caption{\label{fig:figure3} LSR No. 1185: \textit{Schleifer} Ornament in green.}
	\end{figure}
\vfill \break
\clearpage
	
Notice that on Line 27, \verb+#green+ has been added. You can change this to any of the colors listed under "Normal Colors" in the \textit{Notation Reference},\footnote{\url{https://lilypond.org/doc/v2.24/Documentation/notation/list-of-colors}} such as \verb+#'"lightsalmon"+, \verb+#(x11-color "medium turquoise")+, or even \verb+#'"#5e45ad"+.

As an exercise, try replicating the following excerpt:\footnote{See \href{https://github.com/yoshiakionishi/lilypond-snippets/blob/main/DOCUMENT/LSR1185e3.ly}{LSR1185e3.ly} for the answer.}

	\begin{figure}[!htb]
	\centering
\lilypondfile[staffsize=16]{/LSR1185e3.ly}
	\caption{\label{fig:figure4} Can you replicate this?}
	\end{figure}
\hyperref[sec:toc]{\textbf{Table of Contents}}

\vfill \break
\clearpage
%%%%%%%%%%%%%%%%%%%

\section {Example 1: Creating a Time Signature with Its Compound Meter Form}

On January 1, 2025, I came across a post by an anonymous user on Facebook.\footnote{\url{https://www.facebook.com/groups/gnulilypond/posts/10162467719483529/}} The post asked if it would be possible to create a time signature that had its beat structure expressed in the form of a compound meter. Something like this:

\begin{figure}[!htb]
\centering
\lilypondfile[staffsize=16]{/CompoundMeterExample1.ly}
\caption{\label{fig:figure5} What the anonymous user wanted to achieve.}
\end{figure}

I responded to the post with relevant email threads on \verb|lilypond-user| mailing list. I commented that it would be possible to override \verb|TimeSignature.stencil| with custom-made time signatures. Incidentally, I was making a series of \hyperref[sec:time_signatures]{Fractional Time Signatures}, which used this method. 

The code for the aforementioned example is as follows:
\begin{Verbatim}[numbers=left,xleftmargin=5mm]
\version "2.24.4"

{
  \clef "G"
  \time 9/8
  \set beatStructure = #'(2 2 2 3)
  \once \override Timing.TimeSignature.stencil = #ly:text-interface::print
  \once \override Timing.TimeSignature.text = \markup
  {
    \override #'(baseline-skip . 0)
    \center-column \number {9 8}
    \center-column {\fontsize #6 \musicglyph "accidentals.leftparen"}
    \hspace #-0.75
    \override #'(baseline-skip . 0)
    \center-column \number {{2+2+2+3} 8}
    \hspace #-0.75
    \center-column {\fontsize #6 \musicglyph "accidentals.rightparen"}
    \hspace #-1
  }
  \repeat unfold 9 {<e' g'>8 }
}
\end{Verbatim}

I realized that, while this might be an acceptable method if such time signatures appeared only once or twice in a piece, it may become problematic if I had to copy and paste this code every time I have such a time signature. Normally this could easily be resolved by making a variable out of \verb|\override| clauses; however, a piece of music may use time signatures of this form in different configurations, just as the following example:

	\begin{figure}[!htb]
	\centering
\lilypondfile[staffsize=16]{/CompoundMeterExample2.ly}
\label{sec:schemeExample1}
	\caption{\label{fig:figure6} More compound meters.}
	\end{figure}
	
The code:
\begin{Verbatim}[numbers=left,xleftmargin=5mm]
\version "2.24.4"

{
  \clef "G"
  \time 9/8
  \set beatStructure = #'(2 2 2 3)
  \once \override Timing.TimeSignature.stencil = #ly:text-interface::print
  \once \override Timing.TimeSignature.text = \markup
  {
    \override #'(baseline-skip . 0)
    \center-column \number {9 8}
    \center-column {\fontsize #6 \musicglyph "accidentals.leftparen"}
    \hspace #-0.75
    \override #'(baseline-skip . 0)
    \center-column \number {{2+2+2+3} 8}
    \hspace #-0.75
    \center-column {\fontsize #6 \musicglyph "accidentals.rightparen"}
    \hspace #-1
  }
  \repeat unfold 9 {<e' g'>8}

  \time 11/8
  \set beatStructure = #'(2 2 2 3 2)
  \once \override Timing.TimeSignature.stencil = #ly:text-interface::print
  \once \override Timing.TimeSignature.text = \markup
  {
    \override #'(baseline-skip . 0)
    \center-column \number {11 8}
    \center-column {\fontsize #6 \musicglyph "accidentals.leftparen"}
    \hspace #-0.75
    \override #'(baseline-skip . 0)
    \center-column \number {{2+2+2+3+2} 8}
    \hspace #-0.75
    \center-column {\fontsize #6 \musicglyph "accidentals.rightparen"}
    \hspace #-1
  }
  \repeat unfold 11 {<e' g'>8 }

  \time 7/8
  \set beatStructure = #'(2 3 2)
  \once \override Timing.TimeSignature.stencil = #ly:text-interface::print
  \once \override Timing.TimeSignature.text = \markup
  {
    \override #'(baseline-skip . 0)
    \center-column \number {7 8}
    \center-column {\fontsize #6 \musicglyph "accidentals.leftparen"}
    \hspace #-0.75
    \override #'(baseline-skip . 0)
    \center-column \number {{2+3+2} 8}
    \hspace #-0.75
    \center-column {\fontsize #6 \musicglyph "accidentals.rightparen"}
    \hspace #-1
  }
  \repeat unfold 7 {<e' g'>8 }

}
\end{Verbatim}

Writing as long of a code as this (for just three measures!) would be cumbersome, indeed. What could help is to come up with a music function, using the Scheme. 

\subsection{Step 1: Analyze What Could Be Automatized}

I quote the code for the first example of this section again. This time, however, I turn the variables that could change each time I create an instance of this kind of time signature, into \textcolor{red}{red}:

\begin{Verbatim}[numbers=left,xleftmargin=5mm,commandchars={~[]}]
\version "2.24.4"

{
  \clef "G"
  \time ~color[red][9/8]
  \set beatStructure = ~color[red][#'(2 2 2 3)]
  \once \override Timing.TimeSignature.stencil = #ly:text-interface::print
  \once \override Timing.TimeSignature.text = \markup
  {
    \override #'(baseline-skip . 0)
    \center-column \number {~color[red][9 8]~color[black][}]
    \center-column {\fontsize #6 \musicglyph "accidentals.leftparen"}
    \hspace #-0.75
    \override #'(baseline-skip . 0)
    \center-column \number {{~color[red][2+2+2+3]~color[black][}{]~color[red][8]~color[black][}}]
    \hspace #-0.75
    \center-column {\fontsize #6 \musicglyph "accidentals.rightparen"}
    \hspace #-1
  }
  \repeat unfold 9 {<e' g'>8 }
}
\end{Verbatim}



\subsection{Step 2: Write the Code}

It would be good if this function could accept the following as arguments:

\begin{itemize}
\item Time signature of the measure as defined normally in the LilyPond function \verb|\time|. For this, I will set \verb|timesig| as the name of the argument, that tests its value with \verb|fraction?|.
\item The customized stencil of the time signature. I need to declare how it looks, namely:
\begin{itemize}
\item Overall time signature;
\item Numerator portion of the compound meter, and;
\item Denominator portion of the compound meter.
\end{itemize} 
It should look similar to how the LilyPond function \verb|\compoundMeter| that accepts a list of lists.For this, I will set \verb|beatstruct| as the name of the argument, that tests its value with \verb|list?|.
\end{itemize}

I will now build the rest of the function. Notice the way the Scheme code references various locations of a list, using \verb|car|, \verb|cadr|, and so on:

\begin{Verbatim}[numbers=left,xleftmargin=5mm]
\version "2.24.4"

compoundTimeWithBeatStructure =
#(define-music-function (timesig beatstruct) (fraction? list?)
   #{
     \time #timesig
     \set beatStructure = #(cadr beatstruct)
     \once \override Timing.TimeSignature.stencil = #ly:text-interface::print
     \once \override Timing.TimeSignature.text = \markup
     {
       \override #'(baseline-skip . 0)
       \center-column \number
       {
         #(number->string (car (car beatstruct)))
         #(number->string (cadr (car beatstruct)))
       }
       \center-column {\fontsize #6 \musicglyph "accidentals.leftparen"}
       \hspace #-0.75
       \override #'(baseline-skip . 0)
       \center-column \number
       {
         {#(string-join (map number->string (cadr beatstruct)) "+")}
         #(number->string (car (caddr beatstruct)))
       }
       \hspace #-0.75
       \center-column {\fontsize #6 \musicglyph "accidentals.rightparen"}
       \hspace #-1
     }
   #}

   )

{
  \compoundTimeWithBeatStructure 9/8 #'((9 8)(2 2 2 3)(8))
  \repeat unfold 9 {<e' g'>8}
  \compoundTimeWithBeatStructure 11/8 #'((11 8)(2 2 2 3 2)(8))
  \repeat unfold 11 {<e' g'>8}
  \compoundTimeWithBeatStructure 7/8 #'((7 8)(2 3 2)(8))
  \repeat unfold 7 {<e' g'>8}
}
\end{Verbatim}

Thus, there is now a function called \verb|\compoundTimeWithBeatStructure|, whose grammar is:

\begin{Verbatim}
\compoundTimeWithBeatStructure 
	TIME_SIGNATURE #'((TIME_SIGNATURE)(BEAT_STRUCTURE)(DENOMINATOR))
\end{Verbatim}

Running the code will result in the identical snippet as \hyperref[fig:figure6]{the previous figure}:


	\begin{figure}[!htb]
	\centering
\lilypondfile[staffsize=16]{/CompoundMeterExample3.ly}
	\caption{\label{fig:figure7} The same result as before with a shorter code.}
	\end{figure}
	
\par
 

\hyperref[sec:toc]{\textbf{Table of Contents}}

\vfill \break

%%%%%%%%%%%%%%%%%%%
