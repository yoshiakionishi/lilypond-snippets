%%%%%%%%%%%%%%%%%%%
%%%%%%%%%%%%%%%%%%%


\chapter {Spanners}

This chapter covers snippets that take advantages of spanners (text, line, etc.) in one way or another. Because functions such as \verb|\startTextSpan| and \verb|\stopTextSpan| activate and deactivate these snippets, caution must be paid when using more than one of them at the same time. See \hyperref[sec:comb_spanners]{Example in \textbf{Combinations}} to avoid conflicts between or among the spanner snippets.

\section {Grace Note Brackets}
\hfill
\lilypondfile[staffsize=24,hpadding=4,voffset=4pt]{/gracenote_brackets.ly}
\hfill

\subsection{Description}

Replication of grace note brackets seen in scores by Pierre Boulez (e.g. \textit{Sur Incises}\autocite{RN1738}, \textit{...explosante-fixe...}\autocite{RN1737}). Bracket A in the example shows that the grace notes are to be played \textit{before} the beat to which they are applied. Whereas Bracket B shows that the grace notes are to be played \textit{on} the beat to which they are applied.

\subsection{Grammar}
\begin{verbatim}

\graceNoteBeforeBeatOn NOTE
\graceNoteBeforeBeatOff NOTE
\graceNoteAfterBeatOn NOTE
\graceNoteAfterBeatOff NOTE

\end{verbatim}

\subsection{Code}
\begin{verbatim}
\version "2.24.4"

\language "english"

% This code includes snippet for grace note 
% slashes, which has been taken from:
% https://lsr.di.unimi.it/LSR/Item?id=1048


graceNoteBeforeBeatOn =
#(define-music-function (starting_note) (ly:music?)
   #{
     \once \override TextSpanner.style = #'line
     \once \override TextSpanner.bound-details.left.text =
     \markup { \draw-line #'(0 . -1) }
     \once \override TextSpanner.bound-details.right.text =
     \markup {
       \postscript
       "newpath 0 0 moveto 
0 -2.5 rlineto 
stroke 
newpath 
-0.275 -2 moveto 
0.275 -0.75 rlineto 
0.275 0.75 rlineto 
-0.275 -0.2 rlineto 
closepath 
fill"
     }
     \once \override TextSpanner.Y-offset = #5
     \once \override TextSpanner.bound-details.left.padding = #0.5
     \once \override TextSpanner.bound-details.right.padding = #-0.25
     #starting_note
     \startTextSpan
   #})


graceNoteBeforeBeatOff =
#(define-music-function (ending_note) (ly:music?)
   #{
     #ending_note
     \stopTextSpan
   #})


graceNoteAfterBeatOn =
#(define-music-function (starting_note) (ly:music?)
   #{
     \once \override TextSpanner.style = #'line
     \once \override TextSpanner.bound-details.right.text =
     \markup {
       \combine \draw-line #'(0 . -1)
       \postscript "newpath 
0 -1 moveto 
0 -1 rlineto 
stroke"
     }
     \once \override TextSpanner.bound-details.left.text =
     \markup {
       \postscript
       "newpath 0 0 moveto 
0 -1 rlineto 
stroke 
newpath 
-0.275 -0.75 moveto 
0.275 -0.75 rlineto 
0.275 0.75 rlineto 
-0.275 -0.2 rlineto 
closepath 
fill"
     }
     \once \override TextSpanner.Y-offset = #2
     \once \override TextSpanner.bound-details.left.padding = #0.5
     \once \override TextSpanner.bound-details.right.padding = #-0.25
     #starting_note
     \startTextSpan
   #})


graceNoteAfterBeatOff =
#(define-music-function (ending_note) (ly:music?)
   #{
     #ending_note
     \stopTextSpan
   #})

%%%%%%%%%%%%%%%%%%%LSR SNIPPET START%%%%%%%%%%%%%%%%%%

#(define (degrees->radians deg)
   (* PI (/ deg 180.0)))

slash =
#(define-music-function (ang stem-fraction protrusion)
   (number? number? number?)
   (remove-grace-property 'Voice 'Stem 'direction)
   #{
     \once \override Stem.stencil =
     #(lambda (grob)
        (let* ((x-parent (ly:grob-parent grob X))
               (is-rest? (ly:grob?
                          (ly:grob-object x-parent 'rest)))
               (beam (ly:grob-object grob 'beam))
               (stil (ly:stem::print grob)))
          (cond
           (is-rest? empty-stencil)
           ((ly:grob? beam)
            (let* ((refp (ly:grob-system grob))
                 (stem-y-ext (ly:grob-extent grob grob Y))
                 (stem-length
                  (- (cdr stem-y-ext) (car stem-y-ext)))
                 (beam-X-pos (ly:grob-property beam 'X-positions))
                 (beam-Y-pos (ly:grob-property beam 'positions))
                 (beam-slope (/ (- (cdr beam-Y-pos) (car beam-Y-pos))
                              (- (cdr beam-X-pos) (car beam-X-pos))))
                 (beam-angle (atan beam-slope))
                 (dir (ly:grob-property grob 'direction))
                 (line-dy (* stem-length stem-fraction))
                 (line-dy-with-protrusions (if (= dir 1)
                              (+ (* 4 protrusion) beam-angle)
                              (- (* 4 protrusion) beam-angle)))
                 (ang (if (> beam-slope 0)
                          (if (= dir 1)
                              (+ (degrees->radians ang) (* beam-angle 0.7))
                              (degrees->radians ang))
                          (if (= dir 1)
                              (degrees->radians ang)
                              (- (degrees->radians ang) (* beam-angle 0.7)))))
                 (line-dx (/ line-dy-with-protrusions (tan ang)))
                 (protrusion-dx (/ protrusion (tan ang)))
                 (corr (if (= dir 1) (car stem-y-ext) (cdr stem-y-ext))))
            (ly:stencil-add
             stil
             (grob-interpret-markup grob
                               (markup
                                 #:translate
                                 (cons (- protrusion-dx)
                                 (+ corr
                                    (* dir
                                       (- stem-length
                                          (+ stem-fraction protrusion)))))
                                 #:override '(thickness . 1.7)
                                 #:draw-line
                                 (cons line-dx
                                  (* dir line-dy-with-protrusions)))))))
           (else stil))))
   #})

startSlashedGraceMusic = {
  \slash 40 1 0.5
  \override Flag.stroke-style = #"grace"
}
stopSlashedGraceMusic = {
  \revert Flag.stroke-style
}

startAcciaccaturaMusic = {
  \slash 40 1 0.5
  s1*0(
  \override Flag.stroke-style = #"grace"
}
stopAcciaccaturaMusic = {
  \revert Flag.stroke-style
  s1*0)
}
%%%%%%%%%%%%%%%%%%% LSR SNIPPET END %%%%%%%%%%%%%%%%%%


{
  \grace {
    \startSlashedGraceMusic
    \graceNoteBeforeBeatOn d''8^\markup{\box A} e' f'' g' e'' c'
  }
  \graceNoteBeforeBeatOff d'2
  \grace {
    \startSlashedGraceMusic
    \graceNoteAfterBeatOn d''8^\markup{\box B} e' f'' g' e'' c'
  }
  \graceNoteAfterBeatOff d'2
}
\end{verbatim}

\hyperref[sec:toc]{\\ \textbf{Table of Contents}}

\vfill \break

%%%%%%%%%%%%%%%%%%%


\section {Tempo Arrows}
\hfill
\lilypondfile[staffsize=24,hpadding=4,voffset=4pt]{/tempoArrows.ly}
\hfill

\subsection{Description}

Replication of accelerando and rallentando arrows chiefly seen in scores by Tōru Takemitsu.\footnote{Examples abound, but see: \cite{RN1736} and \cite{RN1735} Other composers from the same publishing company, e.g. Toshio Hosokawa, have also adopted variants of the arrows in their music.} The snippets also handle line break.

\subsection{Grammar}
\begin{verbatim}
\accelArrow #Line_angle ... \stopTextSpan
\rallArrow #Line_angle ... \stopTextSpan
\end{verbatim}


\textbf{NB} 
\begin{enumerate}
\item \verb|#Line_angle| sets how angled the horizontal line should be. \verb|#5| should be more than sufficient for a short line.  When it goes over a line break or it extends for a long time, a smaller number may be recommended, such as \verb|#2|. 
\item These commands only set the tempo arrows; as such, indications such as \verb|accel.| and \verb|rall.| need to be added separately. 
\end{enumerate}

\subsection{Code}
\begin{verbatim}
\version "2.24.4"

% freely modified from: https://lsr.di.unimi.it/LSR/Item?id=1168
% as well as http://lsr.di.unimi.it/LSR/Item?id=1023


accelArrow =
#(define-music-function (line_angle) (number?)

   (define x_value (cos (* (/ 3.14159265358979 180) (- 90 line_angle))))
   (define y_value (sin (* (/ 3.14159265358979 180) (- 90 line_angle))))
   #{
     \tweak direction #up
     \tweak style  #'line
     \tweak thickness  #1
     \tweak to-barline ##t
     \tweak rotation #(list line_angle -1 0 )
     \tweak bound-details.left.stencil #ly:text-interface::print
     \tweak bound-details.left.text \markup \postscript
     #(string-append
       "gsave newpath 
0 0 moveto "
       (number->string x_value) " "
       (number->string y_value)
       " rlineto 
stroke 
grestore")
     \tweak bound-details.left-broken.stencil #ly:text-interface::print
     \tweak bound-details.left-broken.text ##f

     \tweak bound-details.right.stencil #ly:text-interface::print
     \tweak bound-details.right.text \markup \postscript
     "newpath 
0 0 moveto 
-1 -0.3 rlineto 
stroke"
     \tweak bound-details.right-broken.stencil #ly:text-interface::print
     \tweak bound-details.right-broken.text ##f
     \tweak font-shape  #'upright
     \tweak bound-details.left.padding  #0
     \tweak bound-details.right.padding  #0
     \tweak breakable ##t
     \tweak after-line-breaking ##t

     \startTextSpan
   #})

rallArrow =
#(define-music-function (line_angle) (number?)

   (define x_value (cos (* (/ 3.14159265358979 180) (- 90 line_angle))))
   (define y_value (sin (* (/ 3.14159265358979 180) (- 90 line_angle))))
   #{
     \tweak direction #up
     \tweak style  #'line
     \tweak thickness  #1
     \tweak to-barline ##t
     \tweak rotation #(list (* -1 line_angle) 1 0 )
     \tweak bound-details.left.stencil #ly:text-interface::print
     \tweak bound-details.left.text \markup \postscript
     #(string-append
       "gsave 
newpath 
0 0 moveto "
       (number->string x_value) " "
       (number->string (* -1 y_value))
       " rlineto 
stroke 
grestore")
     \tweak bound-details.left-broken.stencil #ly:text-interface::print
     \tweak bound-details.left-broken.text ##f

     \tweak bound-details.right.stencil #ly:text-interface::print
     \tweak bound-details.right.text \markup \postscript
     "newpath 
0 0 moveto 
-1 -0.3 rlineto 
stroke"
     \tweak bound-details.right-broken.stencil #ly:text-interface::print
     \tweak bound-details.right-broken.text ##f
     \tweak font-shape  #'upright
     \tweak bound-details.left.padding  #0
     \tweak bound-details.right.padding  #0
     \tweak breakable ##t
     \tweak after-line-breaking ##t

     \startTextSpan
   #})

\score {
  \layout {
    indent = 0
  }
  {
    c'2^\markup{\translate #'(-4 . 2) \box "A"}
    ^\markup {\translate #'(0 . 1.5) \tiny \bold "accel."} 
    	\accelArrow #5   c'2
    c'2 \after 2 \stopTextSpan c'2
    c'2 ^\markup {\translate #'(0 . 1.5) \tiny \bold "rit."}  
    	\rallArrow #3   c'2
    c'2 \after 2 \stopTextSpan c'2 \bar "||"
  }
}

\score {
  \layout {
    indent = 0
    line-width = 40
  }
  {
    c'2^\markup{\translate #'(-4 . 2) \box "B"}
    ^\markup {\translate #'(0 . 1.5) \tiny \bold "accel."} 
    	\accelArrow #5 c'2
    c'2 c'2
    c'2^\markup {\translate #'(0 . 1.5) \teeny \bold "(accel.)"} 
    	\after 2 \stopTextSpan c'2
    c'2 ^\markup {\translate #'(0 . 1.5) \tiny \bold "rit."}  
    	\rallArrow #2 c'2 \break
    c'2^\markup {\translate #'(0 . 1.5) \teeny \bold "(rit.)"} c'2
    c'2 \after 2 \stopTextSpan c'2 \bar "||"
  }
}
\end{verbatim}

\hyperref[sec:toc]{\\ \textbf{Table of Contents}}

\vfill \break

%%%%%%%%%%%%%%%%%%%

