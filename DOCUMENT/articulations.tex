%%%%%%%%%%%%%%%%%%%
%  ARTICULATIONS  %  
%%%%%%%%%%%%%%%%%%%


\chapter {Articulations}

\label{sec:articulations_accent-staccato}

\section {Accent-Staccato}
\hfill
\lilypondfile{/accentStaccato.ly}
\hfill

\subsection{Description}
While \textit{accent-staccato} is not specific to contemporary music, in LilyPond, specifying accent and staccato via \verb|->-.| could cause the two articulation marks to be separated from each other. This happens because of the default setting for accents is to have them placed \textit{outside} the staff line. For example, if you wrote \verb|\stemUp g''->-.|, the following results:

\lilypond[staffsize=30,line-width=1000]{
\version "2.24.4"
\language "english"
{
\stemUp g''->-. 
}
}

This code implements a combination of the music glyphs \verb|scripts.sforzato| and \verb|scripts.staccato| as one entity.

\subsection{Grammar}
\begin{verbatim}
NOTE \accentStaccato
NOTE \accentStaccatoUp
NOTE \accentStaccatoDown
\end{verbatim}
\subsection{Code}
\begin{Verbatim}[numbers=left,xleftmargin=5mm]
\version "2.24.4"

#(append! default-script-alist
          (list
           `(accentStaccatoUp
             . (
                (stencil . ,ly:text-interface::print)
                (text . ,#{ \markup


                            \center-align
                            \combine \musicglyph "scripts.sforzato"
                            \translate #'(0 . -0.75) 
                            \musicglyph "scripts.staccato" #})
                ; any other properties
                (toward-stem-shift-in-column . 1.0)
                (outside-staff-priority . #t)
                (padding . 0.5)
                (avoid-slur . around)
                (direction . ,UP))))

          (list
           `(accentStaccatoDown
             . (
                (stencil . ,ly:text-interface::print)
                (text . ,#{ \markup \center-align
                            \combine \musicglyph "scripts.staccato"
                            \translate #'(0 . -0.75) 
                            \musicglyph "scripts.sforzato" #})
                ; any other properties
                (toward-stem-shift-in-column . 1.0)
                (outside-staff-priority . #t)
                (padding . 0.5)
                (avoid-slur . around)
                (direction . ,DOWN)))))

accentStaccato =  #(make-articulation 'accentStaccatoUp)
accentStaccatoUp = #(make-articulation 'accentStaccatoUp)
accentStaccatoDown = #(make-articulation 'accentStaccatoDown)


{
 \override Staff.TimeSignature.stencil = ##f
 \time 5/4

 c'4\accentStaccato c'4 \accentStaccatoDown c''\accentStaccatoUp
 \stopStaff
 \override Staff.StaffSymbol.line-positions = #'(4 -4)
 \startStaff

 \stemUp g'' \tweak Y-offset #1.5 \accentStaccatoDown
 \stemDown d' \tweak Y-offset #-0.5 \accentStaccatoUp
}
\end{Verbatim}
\hyperref[sec:toc]{\textbf{Table of Contents}}
\vfill \break

%%%%%%%%%%%%%%%%%%%


\section {Accent-Tenuto}
\hfill
\lilypondfile{/accentTenuto.ly}
\hfill

\subsection{Description}
For the same rationale as explained in the \hyperref[sec:articulations_accent-staccato]{Accent-Staccato} entry, this code implements a combination of the music glyphs \verb|scripts.sforzato| and \verb|scripts.tenuto| as one entity.

\subsection{Grammar}
\begin{verbatim}
NOTE \accentTenuto
NOTE \accentTenutoUp
NOTE \accentTenutoDown
\end{verbatim}
\subsection{Code}
\begin{Verbatim}[numbers=left,xleftmargin=5mm]
\version "2.24.4"

#(append! default-script-alist
          (list
           `(accentTenutoUp
             . (
                (stencil . ,ly:text-interface::print)
                (text . ,#{ \markup


                            \center-align
                            \combine \musicglyph "scripts.sforzato"
                            \translate #'(0 . -0.75) 
                            \musicglyph "scripts.tenuto" #})
                ; any other properties
                (toward-stem-shift-in-column . 1.0)
                (outside-staff-priority . #t)
                (padding . 0.5)
                (avoid-slur . around)
                (direction . ,UP))))

          (list
           `(accentTenutoDown
             . (
                (stencil . ,ly:text-interface::print)
                (text . ,#{ \markup \center-align
                            \combine \musicglyph "scripts.tenuto"
                            \translate #'(0 . -0.75) 
                            \musicglyph "scripts.sforzato" #})
                ; any other properties
                (toward-stem-shift-in-column . 1.0)
                (outside-staff-priority . #t)
                (padding . 0.5)
                (avoid-slur . around)
                (direction . ,DOWN)))))

accentTenuto =  #(make-articulation 'accentTenutoUp)
accentTenutoUp = #(make-articulation 'accentTenutoUp)
accentTenutoDown = #(make-articulation 'accentTenutoDown)


{
 \override Staff.TimeSignature.stencil = ##f
 \time 5/4

 c'4\accentTenuto c'4 \accentTenutoDown c''\accentTenutoUp
 \stopStaff
 \override Staff.StaffSymbol.line-positions = #'(4 -4)
 \startStaff

 \stemUp g'' \tweak Y-offset #1.5 \accentTenutoDown
 \stemDown d' \tweak Y-offset #-0.5 \accentTenutoUp
}
\end{Verbatim}
\hyperref[sec:toc]{\textbf{Table of Contents}}
\vfill \break

%%%%%%%%%%%%%%%%%%%


\section {Jeté (Ricochet)}
\hfill
\lilypondfile{/jete_ricochet.ly}
\hfill

\subsection{Description}
I use this notation to designate jeté/ricochet for string instruments, adding that the number of bounces are undetermined.\footnote{Concerning the technique of adding articulation designs to an internal alist, I was inspired by the following thread on \Verb|lilypond-user| mailing list: \url{https://lists.gnu.org/archive/html/lilypond-user/2015-04/msg00105.html}}  

I apply this indication \textit{above} the note regardless of how high or low the note is; however, in case of need, I have supplied the version to be used \textit{under} the note.
\subsection{Grammar}
\begin{verbatim}
NOTE \jete
NOTE \jeteUp
NOTE \jeteDown
\end{verbatim}
\subsection{Code}
\begin{Verbatim}[numbers=left,xleftmargin=5mm]
\version "2.24.4"

jeteDesign =
\markup
\center-align
\combine \combine \combine
\override #'(filled . #t)
\path  #0.1
#'((moveto    -0.25 0.5)
   (curveto   0.35 1.1 0.85 1.1 1.45 0.5)
   (curveto   0.85 0.8 0.35 0.8 -0.25 0.5)
   (closepath))
\draw-circle #0.2 #0 ##t
\translate #'(0.6 . 0) \draw-circle #0.2 #0 ##t
\translate #'(1.2 . 0)\draw-circle #0.2 #0 ##t
#(append! default-script-alist
    (list
     `(jetelistUp
       . (
           (stencil . ,ly:text-interface::print)
           (text . ,#{ \markup \jeteDesign #})
           ; any other properties
           (toward-stem-shift-in-column . 1.0)
           (outside-staff-priority . #t)
           (padding . 0.5)
           (avoid-slur . around)
           (direction . ,UP))))

    (list
     `(jetelistDown
       . (
           (stencil . ,ly:text-interface::print)
           (text . ,#{ \markup \rotate #180 \jeteDesign #})
           ; any other properties
           (toward-stem-shift-in-column . 1.0)
           (outside-staff-priority . #t)
           (padding . 0.5)
           (avoid-slur . around)
           (direction . ,DOWN)))))

jete =  #(make-articulation 'jetelistUp)
jeteUp = #(make-articulation 'jetelistUp)
jeteDown = #(make-articulation 'jetelistDown)


{c'4\jete c'4 \jeteDown c''\jeteUp }
\end{Verbatim}
\hyperref[sec:toc]{\textbf{Table of Contents}}
\vfill \break

%%%%%%%%%%%%%%%%%%%


