
\chapter {Markups}


\section {Conducting Patterns}
\lilypondfile[staffsize=24]{/conducting-patterns.ly}
\hfill

\subsection{Description}
Conducting patterns. While there are several examples of conducting patterns available on LSR,\footnote{See: \url{https://lsr.di.unimi.it/LSR/Item?id=523} and \url{https://lsr.di.unimi.it/LSR/Item?id=259}} the conducting shapes in my implementation are not affected by the horizontal length of given durations.
\subsection{Grammar}
\begin{verbatim}

NOTE \condOne
NOTE \condTwoA
NOTE \condTwoB
NOTE \condThree
NOTE \condDoubleTwoA
NOTE \condDoubleTwoB
NOTE \condDoubleThree

\end{verbatim}
\subsection{Code}
\begin{Verbatim}[numbers=left,xleftmargin=5mm]

condOnePattern =
#'((moveto 0.25 1.75)
   (rlineto 0 -1.75))

condTwoPatternA =
#'((moveto 0.25 1.75)
   (rlineto 0 -1.75)
   (rlineto 2 0)
   (rlineto 0 1.75))

condDoubleTwoPatternA =
#'((moveto 0.25 1.75)
   (rlineto 0 -1.75)
   (rlineto 2 0)
   (rlineto 0 1.75)
   (moveto 0.65 1.75)
   (rlineto 0 -1.35)
   (rlineto 1.2 0)
   (rlineto 0 1.35))

condTwoPatternB =
#'((moveto 0.25 1.75)
   (rlineto 0 -1.75)
   (rlineto 1.25 1.75))

condDoubleTwoPatternB =
#'((moveto 0.25 1.75)
   (rlineto 0 -1.75)
   (rlineto 1.25 1.75)
   (moveto 0.6 1.75)
   (rlineto 0 -0.7)
   (rlineto 0.5 0.7))

condThreePattern =
#'((moveto 1.15 1.75)
   (rlineto -1 -1.75)
   (rlineto 2 0)
   (closepath))

condDoubleThreePattern =
#'((moveto 1.15 1.75)
   (rlineto -1 -1.75)
   (rlineto 2 0)
   (closepath)
   (moveto 1.15 1.05)
   (rlineto -0.385 -0.7)
   (rlineto 0.75 0)
   (closepath))


condOne = ^\markup {
  \override #'(line-join-style . round)
  \path #0.25 #condOnePattern
}

condTwoA = ^\markup {
  \override #'(line-join-style . round)
  \path #0.25 #condTwoPatternA
}
condTwoB = ^\markup {
  \override #'(line-join-style . round)
  \path #0.25 #condTwoPatternB
}
condDoubleTwoA = ^\markup {
  \override #'(line-join-style . round)
  \path #0.25 #condDoubleTwoPatternA
}

condDoubleTwoB = ^\markup {
  \override #'(line-join-style . round)
  \path #0.25 #condDoubleTwoPatternB
}

condThree = ^\markup {
  \override #'(line-join-style . round)
  \path #0.25 #condThreePattern
}

condDoubleThree = ^\markup {
  \override #'(line-join-style . round)
  \path #0.25 #condDoubleThreePattern
}

%% Source inspired by 
%% and adapted from: http://lsr.di.unimi.it/LSR/Item?id=629
spacerVoice = \new Voice {
  \override MultiMeasureRest.transparent = ##t
  \override MultiMeasureRest.minimum-length = #14
  R16*5
}


\score {
  {
    \time 5/8
    b'4 \condTwoA b'4. \condThree \bar "||"
    b'4 \condTwoB b'4. \condThree \bar "||"
    b'8 \condOne b'4 \condTwoA b'4 \condTwoA \bar "||"
    \time 5/16
    << {b'8 \condDoubleTwoA b'8. \condDoubleThree} 
    	\spacerVoice >> \bar "||"
    << {b'8 \condDoubleTwoB b'8. \condDoubleThree} 
    	\spacerVoice >> \bar "||"
  }

}

\end{Verbatim}
\hyperref[sec:toc]{\textbf{Table of Contents}}

\vfill \break




%%%%%%%%%%%%%%%%%%%



\section {Mute Sign}
\lilypondfile[staffsize=24]{/mutesign.ly}
\hfill

\subsection{Description}
Implementation of the mute sign, used to indicate that vibrating strings must be dampened at a specified moment. Its provenance can be traced back to Carlos Salzedo's \textit{Modern Study of the Harp}.\autocite[19]{RN4} 

\subsection{Grammar}
\begin{verbatim}
NOTE/REST^\mutesign
\end{verbatim}
\subsection{Code}
\begin{Verbatim}[numbers=left,xleftmargin=5mm]
mutesign = \markup {
  \translate #'(0.5 . 0)
  \postscript

  "newpath
0.2 setlinewidth
1 setlinecap
0 0 moveto
0 2.5 rlineto
-1.25 1.25 moveto
2.5 0 rlineto
stroke
newpath
0 1.25 0.85 0 360 arc
stroke"

{ c'2. r4^\mutesign }

\end{Verbatim}
\hyperref[sec:toc]{\textbf{Table of Contents}}

\vfill \break


