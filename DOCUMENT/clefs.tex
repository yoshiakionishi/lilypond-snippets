%%%%%%%%%%%%%%%%%%%
%CLEF
%%%%%%%%%%%%%%%%%%%


\chapter {Clefs}

\label{sec:stringPositionClef}
\section {String Position Clef}
\hfill
\lilypondfile[staffsize=24,hpadding=4,voffset=4pt]{/stringPositionClef.ly}
\hfill

\subsection{Description}
String position clef to indicate bowing position. See Discussion for the associated command, \verb|\normalClef|.

\subsection{Grammar}
\begin{verbatim}
\stringPositionClef 
\end{verbatim}
\subsection{Code}
\begin{Verbatim}[numbers=left,xleftmargin=5mm]
stringPositionClefDesign = #(ly:make-stencil (list 'embedded-ps
"gsave
currentpoint translate
/fingboardpath
{
newpath 

-0.55 7.5 moveto
0 -3 rlineto
1 -6.5 rlineto
-1 -1 rlineto
0 -3 rlineto
4.1 0 rlineto
0 3 rlineto
-1 1 rlineto
1 6.5 rlineto
0 3 rlineto
closepath

} def

fingboardpath clip
newpath 
0.15 setlinewidth
0.5 4.75 moveto
0 -6.8 rlineto
-0.75 5 rlineto
3.5 0 rlineto
-0.75 -5 rlineto
0. 6.8 rlineto
stroke
0.35 setlinewidth
-0.4 2.75 moveto
3.75 0 rlineto
stroke

%inner two line
newpath
0.15 setlinewidth
1.16 4.75 moveto
0. -6.8 rlineto
1.8 4.75 moveto
0. -6.8 rlineto
stroke

%bridge
newpath
-0.4 3.6 moveto
0.3 0.4 rlineto
3.2 0 rlineto
0.3 -0.4 rlineto
stroke

%tailpiece
0.15 4.75 moveto 
1 setlinecap
1 setlinejoin
2.75 0 rlineto
-0.65 1.75 rlineto
-0 -0  -0.6 0.55 -1.45 0  rcurveto
closepath
stroke

%mutesign
newpath
0.2 setlinewidth
1 setlinecap
1.5 -2.25 moveto
0 -2.5 rlineto
0.25 -3.5 moveto
2.5 0 rlineto
stroke
newpath
1.5 -3.5 0.85 0 360 arc
stroke
grestore")
	(cons 0 3)
	(cons 0 1))

stringPositionClefSize =
#(lambda (grob)
   (let* ((sPCS (ly:grob-property grob 'font-size 0.0))
          (mult (magstep sPCS)))
     (ly:stencil-scale
      stringPositionClef
      mult mult)))

stringPositionClef = {
  \override Staff.Clef.stencil = \stringPositionClefDesign
}

normalClef = {
  \revert Staff.Clef.stencil
}

{
  \override Staff.StaffSymbol.line-positions = #'(6  -6)
  \override Staff.LedgerLineSpanner.stencil = ##f
  \override Staff.TimeSignature.stencil = ##f
  \override Staff.BarLine.stencil = ##f
  \stringPositionClef c'4 e' g' b' d'' f'' a''
}
\end{Verbatim}

\subsection{Discussion}
\begin{enumerate}
\item With the current design, \verb|c'| would place a note at the lower end of the fingerboard. \verb|a''| would place a note on the same line as the bridge. 

\item The current design comes with the mute sign. If the mute sign is not needed, remove the following portion of the code above:

\begin{Verbatim}[numbers=left,xleftmargin=5mm,firstnumber=64]
%mutesign
newpath
0.2 setlinewidth
1 setlinecap
1.5 -2.25 moveto
0 -2.5 rlineto
0.25 -3.5 moveto
2.5 0 rlineto
stroke
newpath
1.5 -3.5 0.85 0 360 arc
stroke
\end{Verbatim}

\item Once \verb|\stringPositionClef| is used, in order to revert back to the normal clef, \verb|\normalClef| must be used.
\item See \hyperref[sec:comb_strings]{Prescriptive Notation for String Instruments} for a possible use of this clef.
\end{enumerate}

\hyperref[sec:toc]{\textbf{Table of Contents}}

\vfill \break

%%%%%%%%%%%%%%%%%%%

\section {String Position Clef (revised)}
\hfill
\lilypondfile[staffsize=24,hpadding=4,voffset=4pt]{/stringPositionClefRev.ly}
\hfill

\subsection{Description}
This is a revised version of the \hyperref[sec:stringPositionClef]{String Position Clef}, where the fine-tuner pins are graphically represented, as well as the four strings are arranged tighter than the previous version.
\subsection{Grammar}
\begin{verbatim}
\stringPositionClef 
\end{verbatim}
\subsection{Code}
\begin{Verbatim}[numbers=left,xleftmargin=5mm]

% revision june 24 2025

stringPositionClefDesignRev = #(ly:make-stencil (list 'embedded-ps
                                                      "gsave
currentpoint translate
/fingboardpath
{

newpath 
%0 1 .7 0 setcmykcolor 
-0.5 7.5 moveto
0 -3 rlineto
1 -6.5 rlineto
-1 -1 rlineto
0 -3 rlineto
4.05 0 rlineto
0 3 rlineto
-1 1 rlineto
1 6.5 rlineto
0 3 rlineto
closepath

} def

% fingboardpath
 fingboardpath clip

newpath 
0.15 setlinewidth
0.75 5.25 moveto
0 -7.3 rlineto
-0.2 0 rmoveto
-0.75 5 rlineto
3.45 0 rlineto
-0.75 -5 rlineto
-0.2 0 rmoveto
0. 7.3 rlineto
stroke
0.35 setlinewidth
-0.4 2.75 moveto
3.75 0 rlineto
stroke

%inner two lines
newpath
0.15 setlinewidth
1.25 5.5 moveto
0. -7.5 rlineto
1.8 5.5 moveto
0. -7.5 rlineto
stroke


%finetuner pins
0.75 5.4 0.14 0 360 arc
fill
1.25 5.65 0.14 0 360 arc
fill
1.8 5.65 0.14 0 360 arc
fill
2.3 5.4 0.14 0 360 arc
fill



%bridge
newpath
-0.4 3.6 moveto
0.3 0.4 rlineto
3.2 0 rlineto
0.3 -0.4 rlineto
stroke

%tailpiece
0.15 4.75 moveto 
1 setlinecap
1 setlinejoin
2.75 0 rlineto
-0.65 1.75 rlineto
-0 -0  -0.6 0.55 -1.45 0  rcurveto
closepath
stroke

%mute sign, delete if not needed
newpath
0.2 setlinewidth
1 setlinecap
1.5 -2.25 moveto
0 -2.5 rlineto
0.25 -3.5 moveto
2.5 0 rlineto
stroke
newpath
1.5 -3.5 0.85 0 360 arc
stroke

grestore")
                                                (cons 0 3)
                                                (cons 0 1))

strPosClefSize =
#(lambda (grob)
  (let* ((sPCS (ly:grob-property grob 'font-size 0.0))
         (mult (magstep sPCS)))
   (ly:stencil-scale
    strPosClef
    mult mult)))

stringPositionClefDesignRev = {
 \override Staff.Clef.stencil = \stringPositionClefDesignRev
}

normalClef = {
 \revert Staff.Clef.stencil
}

{
 \override Staff.StaffSymbol.line-positions = #'(6  -6)
 \override Staff.LedgerLineSpanner.stencil = ##f
 \override Staff.TimeSignature.stencil = ##f
 \override Staff.BarLine.stencil = ##f
 \stringPositionClefDesignRev c'4 e' g' b' d'' f'' a''
}
\end{Verbatim}

\hyperref[sec:toc]{\textbf{Table of Contents}}

\vfill \break

%%%%%%%%%%%%%%%%%%%



\section {String Position Clef 2}
\hfill
\lilypondfile[staffsize=24,hpadding=4,voffset=4pt]{/stringPositionClef_two.ly}
\hfill

\subsection{Description}
String position clef to indicate bowing position, but this one provides more space between bridge and the edge of the fingerboard, allowing the visual-timbre correspondence between \textit{sul ponticello} and \textit{sul tasto}. 

\subsection{Grammar}
\begin{verbatim}
\stringPositionClef_two
\end{verbatim}
\subsection{Code}
\begin{Verbatim}[numbers=left,xleftmargin=5mm]
\version "2.24.4"

stringPositionClefDesign_two = #(ly:make-stencil (list 'embedded-ps
                                                       "gsave
currentpoint translate
/fingboardpath
{

newpath 
-0.45 4.75 moveto
0 -5 rlineto
0.5 -2.75 rlineto
2.9 0 rlineto
0.5 2.75 rlineto
0 5 rlineto
closepath
} def

fingboardpath clip
newpath 
0.15 setlinewidth
0.5 8 moveto
0 -13.8 rlineto
-0.75 5 rlineto
3.5 0 rlineto
-0.75 -5 rlineto
0 11 rlineto
stroke
0.35 setlinewidth
-0.4 -1 moveto
3.75 0 rlineto
stroke

%inner two line
newpath
0.15 setlinewidth
1.16 4.75 moveto
0. -7.75 rlineto
1.8 4.75 moveto
0. -7.75 rlineto
stroke

%bridge
newpath
-0.4 3.6 moveto
0.3 0.4 rlineto
3.2 0 rlineto
0.3 -0.4 rlineto
stroke

grestore")
	(cons 0 3)
	(cons 0 1))

stringPositionClef_two = {
 \override Staff.Clef.stencil = \stringPositionClefDesign_two
}

normalClef = {
 \revert Staff.Clef.stencil
}

{
 \override Staff.StaffSymbol.line-positions = #'(6  -6)
 \override Staff.LedgerLineSpanner.stencil = ##f
 \override Staff.TimeSignature.stencil = ##f
 \override Staff.BarLine.stencil = ##f
 \stringPositionClef_two c'4^\markup {
  \translate #'(-3 . 2)
  \musicglyph "space"
 }
 _\markup {
  \translate #'(-3 . -3)
  \musicglyph "space"
 }
 e' g' b' d'' f'' a''
}
\end{Verbatim}

\subsection{Discussion}
\begin{enumerate}
\item With the current design, \verb|c'| would place a note at the lower end of the fingerboard. \verb|a''| would place a note on the same line as the bridge. 

\item Once \verb|\stringPositionClef_two| is used, in order to revert back to the normal clef, \verb|\normalClef| must be used.
\end{enumerate}

\hyperref[sec:toc]{\textbf{Table of Contents}}

\vfill \break


%%%%%%%%%%%%%%%%%%%


\label{sec:StringPositionClef3a}
\section {String Position Clef 3a}
\hfill
\lilypondfile[staffsize=24,hpadding=4,voffset=4pt]{/stringPositionClef_three.ly}
\hfill

\subsection{Description}
In contrast to the two types of String Position Clefs introduced earlier, this clef helps facilitate the showing of the positions between the edge of the fingerboard and bridge, as well as between the bridge and the edge of the tailpiece. 

\subsection{Grammar}
\begin{verbatim}
\stringPositionClef_three 
\end{verbatim}
\subsection{Code}
\begin{Verbatim}[numbers=left,xleftmargin=5mm]

stringPositionClef_three_Design = #(ly:make-stencil (list 'embedded-ps
                                                          "gsave
currentpoint translate
/fingboardpath
{
newpath 
0 1 .7 0 setcmykcolor 
0.3 4.75 moveto
0 -4.5 rlineto
-0.2 -0.5 rlineto
0.5 -2.15 rlineto
-1 0 rlineto 
0 -3 rlineto
3.75 0 rlineto
0 3 rlineto
-1 0 rlineto
0.45 2.15 rlineto
-0.15 0.5 rlineto
0 4.5 rlineto
closepath
%stroke
.1 .4 .5 .9 setcmykcolor
} def

%fingboardpath 

fingboardpath clip
newpath 
0.15 setlinewidth
0.8 3.5 moveto
0 -5.85 rlineto
-0.2 0 rmoveto
-0.25 1.3 rlineto
2.2 0 rlineto
-0.2 -1.3 rlineto
-0.2 0 rmoveto
0. 5.85 rlineto
stroke
0.35 setlinewidth
0.15 -1 moveto
3.2 0 rlineto
stroke

%inner two line
newpath
0.15 setlinewidth
1.25 3.6 moveto
0. -5.95 rlineto
1.7 3.6 moveto
0. -5.95 rlineto
stroke

0.8 3.5 0.14 0 360 arc
fill
2.1525 3.5 0.14 0 360 arc
fill
1.25 3.7 0.14 0 360 arc
fill
1.7 3.7 0.14 0 360 arc
fill



%bridge
newpath
0.25 0.6 moveto
0.3 0.4 rlineto
1.85 0 rlineto
0.3 -0.4 rlineto
stroke

%tailpiece
0.425 3 moveto 
1 setlinecap
1 setlinejoin
2.15 0 rlineto
-0.35 1.25 rlineto
-0 -0  -0.65 0.75 -1.55 0  rcurveto
closepath
stroke

%%% mute sign; commentify if not needed %%%
newpath
0.2 setlinewidth
1 setlinecap
1.5 -2.25 moveto
0 -2.5 rlineto
0.25 -3.5 moveto
2.5 0 rlineto
stroke
newpath
1.5 -3.5 0.85 0 360 arc
stroke
%%% end of mute sign for commenting/uncommenting %%%

grestore

")
                                                    (cons 0 3)
                                                    (cons 0 1))

stringPositionClefSize =
#(lambda (grob)
  (let* ((bCS (ly:grob-property grob 'font-size 0.0))
         (mult (magstep bCS)))
   (ly:stencil-scale
    stringPositionClef
    mult mult)))

stringPositionClef_three = {
 \override Staff.Clef.stencil = \stringPositionClef_three_Design
}

{
 \override Staff.StaffSymbol.line-positions = #'(4 0  -4)
 \stringPositionClef_three
 c'4 e' g' b' d'' f'' a''
}
\end{Verbatim}

\subsection{Discussion}
\begin{enumerate}
\item With the current design, \verb|e'| places a note at the lower end of the fingerboard. \verb|b'| places a note at the bridge line, and \verb|f''| places a note on the line indicating the edge of the tailpiece. 

\item The current design comes with the mute sign. If the mute sign is not needed, remove the following portion of the code above:

\begin{Verbatim}[numbers=left,xleftmargin=5mm,firstnumber=64]
%%% mute sign; commentify if not needed %%%
newpath
0.2 setlinewidth
1 setlinecap
1.5 -2.25 moveto
0 -2.5 rlineto
0.25 -3.5 moveto
2.5 0 rlineto
stroke
newpath
1.5 -3.5 0.85 0 360 arc
stroke
%%% end of mute sign for commenting/uncommenting %%%
\end{Verbatim}

\item If you do not wish the ledger lines to appear within the staff line, consider using \verb|\override Staff.NoteHead.no-ledgers = ##t|.
\end{enumerate}

\hyperref[sec:toc]{\textbf{Table of Contents}}

\vfill \break

%%%%%%%%%%%%%%%%%%%

\section {String Position Clef 3b (Longer Span)}
\hfill
\lilypondfile[staffsize=24,hpadding=4,voffset=4pt]{/stringPositionClef_three.ly}
\hfill

\subsection{Description}
The design of this clef is similar to \hyperref[sec:StringPositionClef3a]{String Position Clef 3a}; however, here the distance among the edge of the fingerboard, bridge, and the edge of the tailpiece is set wider. 
\subsection{Grammar}
\begin{verbatim}
\stringPositionClef_three 
\end{verbatim}
\subsection{Code}
\begin{Verbatim}[numbers=left,xleftmargin=5mm]

stringPositionClef_three_longer_Design = #(ly:make-stencil (list 'embedded-ps
                                                          "gsave
currentpoint translate
/fingboardpath
{
newpath 
0 1 .7 0 setcmykcolor 
0.3 5.75 moveto
0 -6.5 rlineto
-0.2 -0.5 rlineto
0.5 -2.15 rlineto
-0.5 0 rlineto 
0 -2.5 rlineto
2.75 0 rlineto
0 2.5 rlineto
-0.5 0 rlineto
0.45 2.15 rlineto
-0.15 0.5 rlineto
0 6.5 rlineto
closepath
%stroke
.1 .4 .5 .9 setcmykcolor
} def

%fingboardpath 

fingboardpath clip
newpath 
0.15 setlinewidth
0.8 4.5 moveto
0 -7.85 rlineto
-0.2 0 rmoveto
-0.25 1.3 rlineto
2.2 0 rlineto
-0.2 -1.3 rlineto
-0.2 0 rmoveto
0. 7.85 rlineto
stroke
0.35 setlinewidth
0.15 -2 moveto
3.2 0 rlineto
stroke

%inner two line
newpath
0.15 setlinewidth
1.25 4.6 moveto
0. -7.95 rlineto
1.7 4.6 moveto
0. -7.95 rlineto
stroke

0.8 4.5 0.14 0 360 arc
fill
2.1525 4.5 0.14 0 360 arc
fill
1.25 4.7 0.14 0 360 arc
fill
1.7 4.7 0.14 0 360 arc
fill



%bridge
newpath
0.25 0.6 moveto
0.3 0.4 rlineto
1.85 0 rlineto
0.3 -0.4 rlineto
stroke

%tailpiece
0.425 4 moveto 
1 setlinecap
1 setlinejoin
2.15 0 rlineto
-0.35 1.25 rlineto
-0 -0  -0.65 0.75 -1.55 0  rcurveto
closepath
stroke

%%% mute sign; commentify if not needed %%%
newpath
0.2 setlinewidth
1 setlinecap
1.5 -3.25 moveto
0 -2.5 rlineto
0.25 -4.5 moveto
2.5 0 rlineto
stroke
newpath
1.5 -4.5 0.85 0 360 arc
stroke
%%% end of mute sign for commenting/uncommenting %%%

grestore

")
                                                    (cons 0 3)
                                                    (cons 0 1))

stringPositionClefSize =
#(lambda (grob)
  (let* ((bCS (ly:grob-property grob 'font-size 0.0))
         (mult (magstep bCS)))
   (ly:stencil-scale
    stringPositionClef
    mult mult)))

stringPositionClef_three_longer = {
 \override Staff.Clef.stencil = \stringPositionClef_three_longer_Design
}

{
 \override Staff.StaffSymbol.line-positions = #'(6 0  -6)
 \stringPositionClef_three_longer
 c'4 e' g' b' d'' f'' a''
}
\end{Verbatim}

\subsection{Discussion}
\begin{enumerate}
\item With the current design, \verb|c'| places a note at the lower end of the fingerboard. \verb|b'| places a note at the bridge line, and \verb|a''| places a note on the line indicating the edge of the tailpiece. 

\item The current design comes with the mute sign. If the mute sign is not needed, remove the following portion of the code above:

\begin{Verbatim}[numbers=left,xleftmargin=5mm,firstnumber=64]
%%% mute sign; commentify if not needed %%%
newpath
0.2 setlinewidth
1 setlinecap
1.5 -3.25 moveto
0 -2.5 rlineto
0.25 -4.5 moveto
2.5 0 rlineto
stroke
newpath
1.5 -4.5 0.85 0 360 arc
stroke
%%% end of mute sign for commenting/uncommenting %%%
\end{Verbatim}

\item If you do not wish the ledger lines to appear within the staff line, consider using \verb|\override Staff.NoteHead.no-ledgers = ##t|.
\end{enumerate}

\hyperref[sec:toc]{\textbf{Table of Contents}}

\vfill \break

%%%%%%%%%%%%%%%%%%%

