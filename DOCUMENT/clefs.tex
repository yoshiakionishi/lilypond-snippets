%%%%%%%%%%%%%%%%%%%
%%%%%%%%%%%%%%%%%%%


\chapter {Clefs}


\section {String Position Clef}
\hfill
\lilypondfile[staffsize=24,hpadding=4,voffset=4pt]{/stringPositionClef.ly}
\hfill

\subsection{Description}
String position clef to indicate bowing position. See Discussion for the associated command, \verb|\normalClef|.

\subsection{Grammar}
\begin{verbatim}
\stringPositionClef 
\end{verbatim}
\subsection{Code}
\begin{Verbatim}[numbers=left,xleftmargin=5mm]
stringPositionClefDesign = #(ly:make-stencil (list 'embedded-ps
"gsave
currentpoint translate
/fingboardpath
{
newpath 

-0.55 7.5 moveto
0 -3 rlineto
1 -6.5 rlineto
-1 -1 rlineto
0 -3 rlineto
4.1 0 rlineto
0 3 rlineto
-1 1 rlineto
1 6.5 rlineto
0 3 rlineto
closepath

} def

fingboardpath clip
newpath 
0.15 setlinewidth
0.5 4.75 moveto
0 -6.8 rlineto
-0.75 5 rlineto
3.5 0 rlineto
-0.75 -5 rlineto
0. 6.8 rlineto
stroke
0.35 setlinewidth
-0.4 2.75 moveto
3.75 0 rlineto
stroke

%inner two line
newpath
0.15 setlinewidth
1.16 4.75 moveto
0. -6.8 rlineto
1.8 4.75 moveto
0. -6.8 rlineto
stroke

%bridge
newpath
-0.4 3.6 moveto
0.3 0.4 rlineto
3.2 0 rlineto
0.3 -0.4 rlineto
stroke

%tailpiece
0.15 4.75 moveto 
1 setlinecap
1 setlinejoin
2.75 0 rlineto
-0.65 1.75 rlineto
-0 -0  -0.6 0.55 -1.45 0  rcurveto
closepath
stroke

%mutesign
newpath
0.2 setlinewidth
1 setlinecap
1.5 -2.25 moveto
0 -2.5 rlineto
0.25 -3.5 moveto
2.5 0 rlineto
stroke
newpath
1.5 -3.5 0.85 0 360 arc
stroke
grestore")
	(cons 0 3)
	(cons 0 1))

stringPositionClefSize =
#(lambda (grob)
   (let* ((sPCS (ly:grob-property grob 'font-size 0.0))
          (mult (magstep sPCS)))
     (ly:stencil-scale
      stringPositionClef
      mult mult)))

stringPositionClef = {
  \override Staff.Clef.stencil = \stringPositionClefDesign
}

normalClef = {
  \revert Staff.Clef.stencil
}

{
  \override Staff.StaffSymbol.line-positions = #'(6  -6)
  \override Staff.LedgerLineSpanner.stencil = ##f
  \override Staff.TimeSignature.stencil = ##f
  \override Staff.BarLine.stencil = ##f
  \stringPositionClef c'4 e' g' b' d'' f'' a''
}
\end{Verbatim}

\subsection{Discussion}
\begin{enumerate}
\item With the current design, \verb|c'| would place a note at the lower end of the fingerboard. \verb|a''| would place a note on the same line as the bridge. 

\item The current design comes with the mute sign. If the mute sign is not needed, remove the following portion of the code above:

\begin{Verbatim}[numbers=left,xleftmargin=5mm,firstnumber=64]
%mutesign
newpath
0.2 setlinewidth
1 setlinecap
1.5 -2.25 moveto
0 -2.5 rlineto
0.25 -3.5 moveto
2.5 0 rlineto
stroke
newpath
1.5 -3.5 0.85 0 360 arc
stroke
\end{Verbatim}

\item Once \verb|\stringPositionClef| is used, in order to revert back to the normal clef, \verb|\normalClef| must be used.
\item See \hyperref[sec:comb_strings]{Prescriptive Notation for String Instruments} for a possible use of this clef.
\end{enumerate}

\hyperref[sec:toc]{\textbf{Table of Contents}}

\vfill \break


