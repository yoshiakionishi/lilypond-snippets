%%%%%%%%%%%%%%%%%%%
%%%%%%%%%%%%%%%%%%%


\chapter {Beams}

\section {Wiggle Beam (zig-zag shaped beam)}
\hfill
\lilypondfile{/wiggleBeam.ly}
\hfill

\subsection{Description}
Ordinary beams are replaced with zig-zag beams. A set of forward then backward beams are printed in the amount specified in the argument. I use this notation in such pieces as \textit{jeux enjeux} (2022) for brass quintet, in order to designate somewhat uneven rhythmic figures, which are nonetheless to be played within the time frame indicated. 

\verb|\wiggleBeamOne| replaces an 8\textsuperscript{th}-note beam.

\verb|\wiggleBeamTwo| replaces a 16\textsuperscript{th}-note beam.

\verb|\wiggleBeamThree| replaces a 32\textsuperscript{nd}-note beam.

\verb|\wiggleBeam_markup| adds a zig-zag beam at will. This allows beaming of mixed note durations, such as:\lilypond{ \layout { 
  \context {
    \Staff
    \remove "Time_signature_engraver"
  }} {c'8 d'16 e'}} \par
\verb|\wiggleBeamStemAdjust| allows the adjustment of a stem length, in the event the wiggle beam and the stem do not touch each other. 

\subsection{Grammar}
\begin{verbatim}
\wiggleBeamOne #vOffset #howMany #width #rotation 
\wiggleBeamTwo #vOffset #howMany #width #rotation 
\wiggleBeamThree #vOffset #howMany #width #rotation 
\end{verbatim}
\textbf{NB} 
\begin{itemize}
\item \verb|hOffset| = (\verb|\wiggleBeam_markup| only) the horizontal offset value originating from where the ordinary beam is placed. 
\item \verb|vOffset| = the vertical offset value originating from where the ordinary beam is placed. 
\item \verb|howMany| = how many "wiggles" to print. \underline{It only accepts integers}.
\item \verb|width| = how wide each "wiggle" should appear. When in doubt, start with \verb|#1|. 
\item \verb|rotation| = a positive value would rotate the beam upward, and the negative value would rotate the beam downward.
\end{itemize}
\hrulefill
\begin{verbatim}
NOTE \wiggleBeam_markup #hOffset #vOffset #howMany #width #rotation 
\end{verbatim}
\textbf{NB} 
\begin{itemize}
\item \verb|hOffset| = the horizontal offset value originating from where the ordinary beam is placed. 
\item \verb|vOffset| = the vertical offset value originating from where an above-staff markup is placed. Thus, \verb|#0| would place a wiggle beam above the staff line. 
\item \verb|howMany| = how many "wiggles" to print. \underline{It only accepts integers}.
\item \verb|width| = how wide each "wiggle" should appear. When in doubt, start with \verb|#1|. 
\item \verb|rotation| = a positive value would rotate the beam upward, and the negative value would rotate the beam downward.
\item More than one \verb|\wiggleBeam_markup| may be added in sequence, provided that for each instance all the arguments are defined.
\end{itemize}
\hrulefill
\begin{verbatim}
\wiggleBeamStemAdjust #fromMiddleLine #howFar NOTE
\end{verbatim}
\textbf{NB} 
\begin{itemize}
\item \verb|fromMiddleLine| = (\verb|\wiggleBeamStemAdjust| only) = determines one end of the stem, \verb|#0| being the middle line of an ordinary 5-line staff. 
\item \verb|howFar| = (\verb|\wiggleBeamStemAdjust| only) = computes how long the stem should be extended. A positive value would draw the stem upward, and a negative value would draw the stem downward. An integer corresponds to the distance between two staff lines of an ordinary 5-line staff.
\end{itemize}

\subsection{Code}
\begin{Verbatim}[numbers=left,xleftmargin=5mm]
wiggleBeamOne =
#(define-music-function (vOffset howMany howWide howTilted)
   (number? number? number? number?) #{
     \once \override Voice.Beam.stencil = #ly:text-interface::print
     \once \override Voice.Beam.text = \markup {
       \translate #(cons 0 vOffset)
       \postscript #(string-append
             "newpath 
              1 setlinejoin 
              1 setlinecap 
              0.35 setlinewidth
              0.13 0 moveto "
             (number->string howMany)
             " {" (number->string (* 0.6 howWide))  " "
             (number->string (+ 0.5 howTilted)) " rlineto "
             (number->string (* 0.6 howWide))
             " -0.5 rlineto} repeat
              stroke"
                   )

     }
   #})

wiggleBeamTwo =
#(define-music-function (vOffset howMany howWide howTilted )
   (number? number? number? number?) #{
     \once \override Voice.Beam.stencil = #ly:text-interface::print
     \once \override Voice.Beam.text = \markup {
       \translate #(cons 0 vOffset)
       \postscript #(string-append
             "newpath 
              1 setlinejoin 
              1 setlinecap 
              0.35 setlinewidth 
              0.13 0 moveto "
             (number->string howMany)
             " {" (number->string (* 0.6 howWide)) " "
             (number->string (+ 0.5 howTilted)) " rlineto "
             (number->string (* 0.6 howWide))
             " -0.5 rlineto} repeat
              stroke newpath 
              0.35 setlinewidth 
              1 setlinejoin 
              0.13 -0.75 moveto "
             (number->string howMany)
             " {" (number->string (* 0.6 howWide))  " "
             (number->string (+ 0.5 howTilted)) " rlineto "
             (number->string (* 0.6 howWide))
             " -0.5 rlineto} repeat
              stroke"
                     )
     }
   #})

wiggleBeamThree =
#(define-music-function (vOffset howMany howWide howTilted )
   (number? number? number? number?)
   #{
     \once \override Voice.Beam.stencil = #ly:text-interface::print
     \once \override Voice.Beam.text = \markup 	{
       \translate #(cons 0 vOffset)
       \postscript #(string-append
             "newpath 
              1 setlinejoin 
              1 setlinecap 
              0.35 setlinewidth 
              0.13 0 moveto "
               (number->string howMany) " {"
               (number->string (* 0.6 howWide))  " "
               (number->string (+ 0.5 howTilted)) " rlineto "
               (number->string (* 0.6 howWide))
             " -0.5 rlineto} repeat 
              stroke 
              newpath 
              0.35 setlinewidth 
              1 setlinejoin 
              0.13 -0.75 moveto "
               (number->string howMany) " {"
               (number->string (* 0.6 howWide))  " "
               (number->string (+ 0.5 howTilted)) " rlineto "
               (number->string (* 0.6 howWide))
               " -0.5 rlineto} repeat 
              stroke 
              newpath 
              0.35 setlinewidth 
              1 setlinejoin 
              0.13 -1.5 moveto "
               (number->string howMany) " {"
               (number->string (* 0.6 howWide)) " "
               (number->string (+ 0.5 howTilted)) " rlineto "
               (number->string (* 0.6 howWide))
               " -0.5 rlineto} repeat 
              stroke"
                     )
     }
   #})

wiggleBeam_markup =
#(define-music-function (hOffset vOffset howMany howWide howTilted )
   (number? number? number? number? number?)
   #{
     ^\markup 	{
       \translate #(cons hOffset vOffset)
       \postscript #(string-append
             "newpath 
              1 setlinejoin 
              1 setlinecap 
              0.35 setlinewidth 
              0.17 0 moveto "
             (number->string howMany) " {"
             (number->string (* 0.6 howWide))  " "
             (number->string (+ 0.5 howTilted)) " rlineto "
             (number->string (* 0.6 howWide))
             " -0.5 rlineto} repeat 
              stroke"
                     )

     }
   #})

wiggleBeamStemAdjust =
#(define-music-function (fromMiddleLine howFar)
   (number? number?)
   #{
     \once \override Stem.stencil = #ly:text-interface::print
     \once \override Stem.text = \markup {
       \postscript #(string-append
             "newpath 
              0.12 setlinewidth 
              0 " (number->string fromMiddleLine) " moveto 
              0 " (number->string howFar) " rlineto 
              stroke"
                     )
     }
   #})
   
{
  \wiggleBeamTwo #0 #9 #1.01 #0 c'16 c'
  \wiggleBeamStemAdjust #-3 #3.4 c' c'
  \wiggleBeamTwo #0 #5 #1.82 #0 g''
  \wiggleBeamStemAdjust #2.5 #-3 g''
  \wiggleBeamStemAdjust #2.5 #-3 g'' g''
  \wiggleBeamTwo #-1 #9 #1.01 #-0.15 f''
  \wiggleBeamStemAdjust #1.5  #-3.5 e''
  \wiggleBeamStemAdjust #1 #-3.5 d''
  \wiggleBeamStemAdjust #0.5 #-3.5 c''
  \wiggleBeamOne #-3.5 #5 #1.4 #0.15 b'8 
  c''16 \wiggleBeam_markup #0 #-4.8 #2 #1.4 #0.15 d''
  \wiggleBeamThree #-1.3 #19 #0.73 #0  g''32
  \wiggleBeamStemAdjust #1.5  #-4 e''
  \wiggleBeamStemAdjust #0.5  #-3 c'' g'' e''
  \wiggleBeamStemAdjust #0.5  #-3  c''
  \wiggleBeamStemAdjust #2.5  #-5 g'' e''
  \bar ".."
}
\end{Verbatim}

\subsection{Discussion}
\begin{enumerate}
\item Admittedly, while the current setup allows great flexibility in making the wiggle beams appear, it is entirely possible that some of the parameters be automated. 
\item When using many wiggle beams, it may be easier to make the score proportionally notated, in order to avoid the micromanagement of the parameters. 
\end{enumerate}
\hyperref[sec:toc]{\textbf{Table of Contents}}
